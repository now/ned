\startenvironment include/layout

\startsectionblockenvironment[frontpart]
  \setuppagenumbering[conversion=romannumerals]
\stopsectionblockenvironment

\startsectionblockenvironment[bodypart]
  \setuppagenumber[number=3]
\stopsectionblockenvironment

% TODO: is it better at none?
\setupwhitespace[none]
\setupindenting[medium]
\setupspacing[broad]

\installlanguage
  [en]
  [spacing=broad,
   leftsentence={\hbox{~--~}},
   rightsentence={\hbox{~--~}},
   leftsubsentence={--},
   rightsubsentence={--},
   leftquote=\upperrightsinglesixquote,
   rightquote=\upperrightsingleninequote,
   leftquotation=\upperrightdoublesixquote,
   rightquotation=\upperrightdoubleninequote,
   date={month,\ ,day,{,\ },year},
   state=stop]

\mainlanguage
  [en]

\setuppagenumbering[alternative=doublesided,location=]
%\setupheads[alternative=inmargin]
%\setuphead[chapter][style=\bfc]
%\setuphead[section][style=\bfb]
%\setuphead[subsection][style=\bfa]

\setupcaptions[width=\textwidth,align=middle]

\setupcolors[state=start]
%\definecolor[activetrans][r=.1, g=.5, b=.8]
%\definecolor[activetrans][r=1, g=.6, b=.6]
%\definecolor[activetrans][r=.6, g=.8, b=1]
\definecolor[activetrans][r=.286275,g=.729412,b=.25098]
\definecolor[activestate][activetrans]
% XXX: use this for various titles and such.
\definecolor[titlecolor][r=.388235,g=.541176,b=.686274]
\definecolor[titlegraphic][r=.8,g=.8,b=.8]


\setuphyphenmark
  [sign={-}]

%\starttypescript [serif] [default] [size]
%  \definebodyfont
%  [4pt,5pt,6pt,7pt,8pt,9pt,9.5pt,10pt,11pt,12pt,14.4pt,20.7pt]
%  [rm] [default]
%\stoptypescript


% TODO: enable when producing screen document
%\setupinteraction
%  [state=start]

% TODO: choose one...
%\usetypescript[fourier][ec]
%\setupbodyfont[fourier]
%\usetypescript[ant][ec]
%\setupbodyfont[ant]
%\usetypescript[palatino][ec]
%\setupbodyfont[palatino]

%\usetypescript [modern][\defaultencoding]

\definetypeface[boldmath][mm][boldmath][computer-modern][computer-modern][encoding=default]

%\setupbodyfont[modern]


\setuppublications[alternative=weibull]

\setupsynonyms
  [abbreviation]
  [textstyle=smallcaps,
   criterium=all]

% TODO: 
\setupcapitals
  [title=yes,
   sc=yes]

\setupformulas[way=bychapter]

\setupcombinedlist
  [content]
  [alternative=a,
   interaction=all,
   level=section]

\setuplist
  [chapter]
  [numbercolor=red,
   style=bold] % numberstyle=mediaeval,

\setuplist
  [section]
  [alternative=d,
   margin=2em,
   pagestyle=slanted]
%  [alternative=b,before=\blank,after={\vskip.7\baselineskip},style=bold]

%\setuplist
%  [section,subsection]
%  [margin=2em,distance=2em]

\startuniqueMPgraphic{FootnoteRule}{amplitude,color}
  numeric amp; amp := \MPvar{amplitude};
  vardef sine(expr x) =
    amp * sind(x * ((3.14159 * 2) / 5) * (360 / (3.14 * 2)))
  enddef;
  def sinewave(expr cycles) =
    ( origin{curl 0}
      for x = 1 upto (cycles * 5 - 1):
        .. (x,sine(x))
      endfor
      .. {curl 0}((cycles * 5),sine(cycles * 5)) )
  enddef;
  draw sinewave(20) scaled 1pt withcolor \MPvar{color};
\stopuniqueMPgraphic

\setupMPvariables
  [FootnoteRule]
  [amplitude=0.5,
   color=titlecolor]

\def\FootnoteRule%
  {\strut\reuseMPgraphic{FootnoteRule}\vskip0pt}

%\def\FootNoteLine%
%  {\strut\vrule height .4pt depth 0pt width 9.25pc
%   \vskip0pt}

\setupfootnotes
  [rule=\FootnoteRule,
   before=\blank,
   way=bychapter]

\definetext[chapter][footer][pagenumber]
% TODO: is this better than setting header=high?
\definetext[chapter][header][]
\setuphead[chapter][header=chapter,footer=chapter,page=right,prefix=+]

\setupheadertexts
  [section][pagenumber]
  [pagenumber][chapter]

\setupregister
  [index]
  [indicator=no]

\definecolor[lightgray][s=.96]

\setupframedtexts
  [background=color,
   backgroundcolor=lightgray,
   frame=off,
   rightframe=on,
   framecolor=titlecolor,
   rulethickness=3pt]
% TODO: anna
%   framecolor=darkgreen,

\setupurl
  [color=blue]

\setuptyping
  [file]
  [numbering=line]

\defineframedtext
  [shellsessionframe]
  [background=color,
   backgroundcolor=lightgray,
   frame=off,
   width=broad]
\setuptyping
  [shellsession]
  [before=\startshellsessionframe,
   after=\stopshellsessionframe]

% TODO: how about indentext here?
\setuptyping
  [C]
  [bodyfont=10pt,
   indentnext=no]

\definetyping[Ruby]
\setuptyping
  [Ruby]
  [bodyfont=10pt,
   indentnext=no]

\setupFLOWshapes[background=color,backgroundcolor=lightgray]

\setupitemgroups
  [indentnext=no,
   headstyle=bold]

\defineitemgroup
  [propertylist]

\setuppropertylist
  [each]
  [packed]
  [symcolor=titlecolor]

\defineitemgroup
  [enumerate]

\setupenumerate
  [each]
  [m,packed]
  [placestopper=no]

\setupenumerate
  [1]
  [m,packed,inmargin]

\setupitemize
  [each]
  [packed]

\setupitemize
  [1]
  [inmargin,packed]

%\setupitemize
%  [each]
%  [inmargin,packed]
%  [symcolor=titlecolor]
%
%\setupitemize
%  [1]
%  [inmargin,packed]
%
%\setupitemize
%  [2]
%  [inmargin,packed]

\definesymbol[1][{\color[titlecolor]{\mathematics{\cdot}}}]
\definesymbol[2][{\color[titlecolor]{\symbol[dash]}}]

\setupalgorithmio
  [each]
  [5*broad,
   packed]
  [symstyle=slanted]

\setupfloats
  [indentnext=no]

\setupcaptions
  [headstyle=\bf\x,
   style=\x]

\setupcaption
  [describedfigure]
  [align=right]

\setupformulas
  [indentnext=no]

\setupdescriptions
  [notation,
   definition]
  [indentnext=no]

\setdigitmode 4
\digittemplate 1,000,000.00

\unexpanded\def\dodowncase{\lowercase}

\definelayer
  [dedication]
  [width=\makeupwidth,
   height=\textheight]

\setlayerframed
  [dedication]
  [x=\makeupwidth,
   y={.5\textheight},
   location={left,bottom}]
  [frame=off,
   offset=none,
   align={right,lohi}]
  {\hbox to \hsize {\hss\ss\sl You know who you are: Thanks}\space}

\setuphead
  [chapter]
  [placehead=empty,
   header=chapter,
   style=\scb, % BUG: should be textstyle
   textcommand=\dodowncase,
   numberstyle=mediaeval]

\definetext
  [chapter]
  [header]
  [\setups{chapter}]
  []

\startsetups chapter
  \setups[chapter:title]
  \setups[chapter:number]
  \setups[chapter:finish]
\stopsetups

\definelayer
  [chapter]
  [width=\dimexpr(\makeupwidth+\cutspace),
   height=\headerheight]

\startsetups chapter:title
  \setlayerframed
    [chapter]
    [x=\dimexpr(\makeupwidth),%+\cutspace),
     location={left,bottom}]
    [height=\headerheight,
     foregroundcolor=black,
     frame=off,
     offset=none,
     align={right,lohi}]
    {\hbox to \hsize % spread .5\cutspace
       {\hss
        \doiftextelse{\placeheadtext[chapter]}%
          {\placeheadtext[chapter]}%
          {\placeheadtext[title]}
        \hss}\space
     \vskip.5cm\hrule}% \hrule
\stopsetups

\startsetups chapter:number
  \setlayerframed
    [chapter]
    [x=\dimexpr(\makeupwidth+\cutspace),
     y=\vsize,
     location={left,bottom}]
    [width=\dimexpr(\cutspace-\rightmargindistance),
     height=\dimexpr(\cutspace-\rightmargindistance),
     foregroundcolor=titlecolor,
%     background=color,
%     backgroundcolor=gray,
     frame=off,
     offset=none,
     align={middle,lohi}]
    {\hbox to \hsize
       {\hss
        \doifmode{*bodypart}
          {\NormalizeFontWidth
           \NumberFont
           {8}
           {\dimexpr(\hsize-1.6cm)\relax}
           {RegularBold}
           \NumberFont\headnumber[chapter]}%
%           \bfd\setupinterlinespace%
%           \ss\headnumber[chapter]}% % use \headnumber[chapter]
        \hss}}
\stopsetups

\startsetups chapter:finish
  \framed
    [width=\makeupwidth,
     height=\headerheight,
     background=chapter,
     frame=off]
    {}
\stopsetups

%   textcommand=\dodowncase,
\setuphead
  [section]
  [style=smallcaps,%{\setsmallbodyfont\sc},
%   numberstyle={\setsmallbodyfont},
   before=\blank,
   color=titlecolor,
   textcommand=\dodowncase,
   after=\blank]

% TODO: what do we use?
\def\SubSectionCommand#1#2%
  {\vbox{#1\quad#2}}
\setuphead
  [subsection]
  [style=slanted,
   textcommand=,
% textstyle={\setsmallbodyfont\sl},
   numberstyle=normal,
   color=titlecolor,
   before=\blank,
   after=\blank,
   command=\SubSectionCommand]

%\starttypescript [map] [latin-modern-os] [ec,texnansi,qx,t5,pl0,il2]
%  \loadmapfile[\typescriptthree-os-public-lm.map]
%\stoptypescript
%\usetypescript [modern][\defaultencoding]
%\usetypescript [map] [latin-modern-os] [\defaultencoding]
%
%\setupbodyfont
%  [modern]

% HACK: a bad hack to get numbers in headers to display in mediaeval
\def\Mydoseparatednumber#1.#2%
  {{\os #1}%
   \ifx#2\empty
     \expandafter\gobbleuntilrelax
   \else \numberseparator
     \expandafter\Mydoseparatednumber
   \fi{\os #2}}

\def\Myseparatednumber  #1{\Mydoseparatednumber  #1.\empty\relax}
\def\dohandleheadnumber#1%
  {\expanded{\Myseparatednumber{#1}}}

%\usetypescript [modern][texnansi]
%\loadmapfile[texnansi-oslftt-public-lm]
%\setupbodyfont[modern]

\stopenvironment
