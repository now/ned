\startcomponent thesis/chapters/pattern-matching

\project masters-project
\product thesis

% TODO: ask anna
\chapter
  [patterns]
  {Pattern Matching}
%  {Pattern Matching Using Regular Expressions}
%  {Pattern Matching: Finding Patterns in Text}


Pattern matching is an important problem in computer science, put to good use
in data processing, lexical analysis, information retrieval, and text editing.
Text editors often allow the user to search for patterns in the text currently
being edited.  Database query languages|<|such as \SQL\ \cite[SQL03]|>|allow
complex queries to use patterns for selecting relevant data.  Many programming
languages|<|such as \Perl\ \cite[Perl00], \CRM\ \cite[CRM04], and \AWK\
\cite[AWK88]|>|make extensive use of pattern matching, as their intent is
generally text transformation or information extraction of some kind.

The general form of the pattern matching problem is to find an occurrence of
some pattern $p$ as a substring of a string.  We will refer to this string as
the \DefineTerm{input}.  If we let $p$ be \TypedRegex{here} and the input be
\TypedString{here or there}, then $p$ will match two substrings of the input:
\TypedString{\REMatch{here} or t\REMatch{here}}.

\placefigure
  [here]
  [figure:pattern matching model]
  {A model for pattern matching}
  {\FLOWchart[pattern matching model]}

\infigure[figure:pattern matching model] shows a model for pattern matching.
The model’s input is some string $s$, again, referred to as the \Term{input},
and a pattern $p$.  From the pattern we generate a pattern matcher $M$, which
will then seek the input for matches of it, giving either “yes” or “no” as its
output depending on whether the input contains a substring matching the pattern
or not.

The actual input and output may vary widely depending on the application.  For
many software tools the input is a file consisting of lines of text plus a
pattern to match and the output is a set of lines matching the pattern, perhaps
modified in some way.  For database systems, the input is a table of data plus
a query describing a pattern to match against the table and the output is a set
of rows matching this pattern.  In a text editor the input is the text
currently being edited plus some pattern identifying a position in the text to
be edited and the output is an updated display of the text.

The rest of this chapter will discuss pattern matching using {\em regular
expressions} as the input pattern and {\em finite automata} as the generated
pattern matcher.


\section{Pattern Matching Using Regular Expressions}

It’s now time to explain how regular expressions can be applied to the task of
pattern matching.

As we saw in \inchapter[definitions], a regular expression is a compact way of
describing a regular language.  We also saw that we can generate a finite
automaton that will accept precisely the set of strings in such a language,
using the regular expression as a specification.  Thus, what we do to use
regular expressions in our pattern||matching model is to write a regular
expression $r$ and feed it as the input pattern $p$.  We then generate a finite
automaton $A$ from this regular expression and feed it the input $s$.
Automaton $A$ will then either \Term{accept} or \Term{reject} it, depending on
whether $s$ is a part of the language described by $r$ or not.

The output from this procedure is often interpreted further.  The command
\Command{egrep}, for example, reports lines in files matching $r$ by augmenting
$r$ with the prefix regular expression \TypedRegex{.*}, thus allowing it to
match any prefix of the input and then generating an \NFA|<|or sometimes a
\DFA|>|$A$ from \TypedRegex{.*}$r$.  Finally, it breaks up the file into the
lines constituting it by looking for line||ending characters, such as
\CodePoint{000A} (\CharacterName{line feed}) and passing them, in turn, to $A$.
If $A$ \Term{accepts} the line, \Command{egrep} outputs it to the user along
with the name of the file in which it was found.  An example of using
\Command{egrep} may clarify matters.

\startexample[example:spell-checking with egrep]
  Say that we have written a couple of manuscripts and have since found that we
  have misspelled the words “acceptable” and “color”|<|writing “acceptible” and
  “colour” in their place|>|in at least one of them.  We don’t want to read
  through all of the manuscripts to check for the occurrences of the same
  misspellings, so what do we do?  Regular expressions to the rescue!
  Notably|<|as we are working on a \UNIX\ system|>|we will use the \UNIX\
  command \Command{egrep} for solving this task.

  As noted in \inintermezzo[the real world:intermezzo:grep and egrep],
  \Command{egrep} allows us to specify a regular expression that matches the
  strings that we wish to find, and the files whose contents we would like to
  search, so let’s try it.  Before we write our command, however, let’s take a
  look at the files that we’re about to search:

  \startitemize
    \head \FileName{annual-report.txt} \par
    \typefile{../thesis/data/annual-report.txt}

    \head \FileName{brown-commentary.txt} \par
    \typefile{../thesis/data/brown-commentary.txt}

    \head \FileName{spelling.txt} \par
    \typefile{../thesis/data/spelling.txt}
  \stopitemize

  Perhaps if we’d written \FileName{spelling.txt} before the other two, we
  wouldn’t be in this mess.  Anyway, now what we do is invoke \Command{egrep}
  in the following manner:

  \startshellsession
% egrep --line-number 'acceptible|colour' \
> annual-report.txt brown-commentary.txt spelling.txt
annual-report.txt:2:acceptable amount of our multi-coloured
brown-commentary.txt:2:unacceptible in my humble opinion.  It is,
spelling.txt:3:_not_ "acceptible" and "colour",
spelling.txt:6:find "acceptible" acceptable.
  \stopshellsession

  We invoked \Command{egrep} with the \type{--line-number} option, which tells
  \Command{egrep} to print out the line number along with the line of text and
  name of the file in which it was found.  \Command{egrep} treats each line of
  each file as a separate entity, and checks if any substring of it is accepted
  by the finite automaton generated from the given regular expression.  If we
  look at the output from \Command{egrep} presented above, we see that the
  second line of \FileName{annual-report.txt} contains a misspelling of {\em
  color}.  Notice how \Command{egrep} matched the regular expression against
  that line even though the actual matching string \TypedString{colour} was
  surrounded by text both before and after, and was even part of a longer
  word---\TypedString{coloured}.  The same can be said about the second line of
  \FileName{brown-commentary.txt}.  \Command{egrep} matched
  \TypedString{acceptible}, even when it was a proper suffix of the word
  \TypedString{unacceptible}.  There are ways to tell \Command{egrep} to only
  match whole words so that this wouldn’t have happened.  In our case,
  however, it was the right thing, as it caught two misspellings that we hadn’t
  initially considered.  We ignore the misspellings in \FileName{spelling.txt},
  as their presence is acceptable.  So now we know where we can find our
  misspelled words and we can go ahead and amend our erroneous manuscripts.
\stopexample

As displayed in \inexample[example:spell-checking with egrep], we generally
don’t force any specific structure on the data we search, except that some
tools|<|such as \Command{egrep} and \Command{awk} \cite[AWK88]|>|treat textual
lines as separate units.  We’ll see later that this is often an unnecessary
restriction that can lead to certain difficulties.


\section
  [submatch addressing]
  {Submatch Addressing}

As discussed in the previous section, we match a regular expression against
some input, ignoring its structure, instead looking for substrings within the
input that are accepted by the finite automaton that’s been compiled from the
regular expression.  When performing pattern matching in this way, it’s often
desirable to know what parts of a regular expression matched what parts of the
input.  This task is known as \DefineTerm{submatch addressing} and it’s
performed by altering the semantics of our regular expressions slightly.

The subexpression grouping operator, $(\ldots)$, is modified in such a way,
that whatever matched by the contained expression will be recorded for future
reference.  We record this information by keeping a pair of indexes
\SubMatch{b}{e} into the input, where $b$ is the index of the first character
matched by the subexpression and $e$ is the index of the last character matched
by the subexpression plus one.  The first index in a string is $0$.  The length
of the match can easily be computed from this information; it’s simply $e - b$.
The pair \SubMatch{b}{e} is known as a \DefineTerm{submatch address}.

As an example, consider the regular expression
\TypedRegex{(this|that) or (that|this)} applied to the input
\TypedString{this or that}, which is twelve characters long.  The regular
expression will match the whole input, \TypedString{\REMatch{this or that}},
and its submatch address will be \SubMatch{0}{12}.  The regular expression also
has two subexpressions that have been marked for submatch addressing,
\TypedRegex{(this|that)} and \TypedRegex{(that|this)}.  The first will match
\TypedString{\REMatch{this} or that}, giving us the submatch address
\SubMatch{0}{4}.  The second will match \TypedString{this or \REMatch{that}},
giving us the submatch address \SubMatch{8}{12}.


\section{Resolving Submatch Addressing Ambiguities}

Submatch addressing doesn’t come without its own set of subtleties; one of
them being the fact that a subexpression may participate in the match of
several different substrings of the input.  We must therefore define a set of
rules that remove the possibility of ambiguity.  Let’s first look at an
example of a regular expression where submatch addressing has several possible
solutions.

\startexample[example:submatch addressing ambiguities]
  Apply the regular expression \TypedRegex{(a*)(a*)} on input
  \TypedString{aaa}.  \intable[table:ambiguous matches] lists all possible
  submatches for the two subexpressions of the given regular expression.

  \placetable
    [here]
    [table:ambiguous matches]
    {Possible submatches for \TypedRegex{(a*)(a*)} on input \TypedString{aaa}}
    \starttable[|r|c|c|r|c|c|]
    \HL
    \NC \bf \# \NC \bf First \TypedRegex{(a*)} \NC \bf Second \TypedRegex{(a*)}
    \VC \bf \# \NC \bf First \TypedRegex{(a*)} \NC \bf Second \TypedRegex{(a*)}
    \NC\AR
    \HL
    \NC 1 \NC \SubMatch{0}{0} \NC \SubMatch{0}{0}
       \VC 11 \NC \SubMatch{1}{1} \NC \SubMatch{1}{1} \NC\AR
    \NC 2 \NC \SubMatch{0}{0} \NC \SubMatch{0}{1}
       \VC 12 \NC \SubMatch{1}{1} \NC \SubMatch{1}{2} \NC\AR
    \NC 3 \NC \SubMatch{0}{0} \NC \SubMatch{0}{2}
       \VC 13 \NC \SubMatch{1}{1} \NC \SubMatch{1}{3} \NC\AR
    \NC 4 \NC \SubMatch{0}{0} \NC \SubMatch{0}{3}
       \VC 14 \NC \SubMatch{1}{2} \NC \SubMatch{2}{2} \NC\AR
    \NC 5 \NC \SubMatch{0}{1} \NC \SubMatch{1}{1}
       \VC 15 \NC \SubMatch{1}{2} \NC \SubMatch{2}{3} \NC\AR
    \NC 6 \NC \SubMatch{0}{1} \NC \SubMatch{1}{2}
       \VC 16 \NC \SubMatch{1}{3} \NC \SubMatch{3}{3} \NC\AR
    \NC 7 \NC \SubMatch{0}{1} \NC \SubMatch{1}{3}
       \VC 17 \NC \SubMatch{2}{2} \NC \SubMatch{2}{2} \NC\AR
    \NC 8 \NC \SubMatch{0}{2} \NC \SubMatch{2}{2}
       \VC 18 \NC \SubMatch{2}{2} \NC \SubMatch{2}{3} \NC\AR
    \NC 9 \NC \SubMatch{0}{2} \NC \SubMatch{2}{3}
       \VC 19 \NC \SubMatch{2}{3} \NC \SubMatch{3}{3} \NC\AR
    \NC 10 \NC \SubMatch{0}{3} \NC \SubMatch{3}{3}
       \VC 20 \NC \SubMatch{3}{3} \NC \SubMatch{3}{3} \NC\AR
    \HL
    \stoptable
  
  For this particular instance of the problem, there are

    \[\sum_{0 \leq i \leq j \leq k \leq 3} 1 = \binom 6 3\]

  possible pairs; and in general there are

    \[\binom{n + k}{k},\]

  for $n$ \type{a}s and $k$ \TypedRegex{(a*)}||subexpressions.
\stopexample

As \inexample[example:submatch addressing ambiguities] shows, we have a lot of
possible submatches to choose from; we would prefer to have only one.
Therefore, we define the following three rules that serve to identify a unique
solution, in order of decreasing priority:

\startenumerate
  \head[leftmost longest rule] Leftmost Longest Rule

  If a regular expression can match more than one substring of the input, the
  match starting earliest is chosen.  If more than one substring begins at that
  position, the longest one among them is chosen.

  \head[subexpression rule] Subexpression Rule

  Subexpressions abide by rule \in[leftmost longest rule].  Subexpressions
  starting earlier in the regular expression take priority over ones starting
  later.  Matching an empty substring is considered longer than matching no
  substring at all.

  \head Repeated Matching Rule

  If a subexpression matches more than one substring of the whole match, the
  last such substring is chosen.  Note that the candidate substrings won't
  overlap.
\stopenumerate

\inexamples[example:leftmost longest] \andexample[example:subexpression] show
how these rules apply to regular expressions.

\startexample[example:leftmost longest]
  Consider again the regular expression of
  \inexample[example:submatch addressing ambiguities], \TypedRegex{(a*)(a*)}.
  Rule \in[leftmost longest rule] tells us that the whole
  input|<|\TypedString{aaa}|>| must be matched.  This cuts down the result set
  to that listed in \intable[table:leftmost cutdown].

  \placetable
    []
    [table:leftmost cutdown]
    {Leftmost longest submatches submatches for \TypedRegex{(a*)(a*)} on input
     \TypedString{aaa}.}
    \starttable[|c|c|]
    \HL
    \NC \bf First \TypedRegex{(a*)} \NC \bf Second \TypedRegex{(a*)} \NC\AR
    \HL
    \NC \SubMatch{0}{0}             \NC \SubMatch{0}{3}               \NC\AR
    \NC \SubMatch{0}{1}             \NC \SubMatch{1}{3}               \NC\AR
    \NC \SubMatch{0}{2}             \NC \SubMatch{0}{3}               \NC\AR
    \NC \SubMatch{0}{3}             \NC \SubMatch{3}{3}               \NC\AR
    \HL
    \stoptable

  Rule \in[subexpression rule] then tells us to choose the submatch addresses
  of the last row, since the first subexpression has the longest match in that
  row.
\stopexample

\startexample[example:subexpression]
  Consider the regular expression \TypedRegex{(a|aa)*}.
  \intable[table:a or aa matches] shows what the different parts of the regular
  expression match.

  \placetable
    []
    [table:a or aa matches]
    {Submatch addressesing for \type{(a}{\tttf\char`\|}\type{aa)*} on various
    input}
    \starttable[|l|c|c|c|c|]
    \HL
    \NC \bf Input         \NC \TypedRegex{(a|aa)*}  \NC \TypedRegex{a|aa}
    \NC \TypedRegex{a}    \NC \TypedRegex{aa}       \NC\AR
    \HL
    \NC $ε$               \NC \SubMatch{0}{0}       \NC \SubMatch{-1}{-1}
    \NC \SubMatch{-1}{-1} \NC \SubMatch{-1}{-1}     \NC\AR
    \NC \TypedString{a}   \NC \SubMatch{0}{1}       \NC \SubMatch{0}{1} 
    \NC \SubMatch{0}{1}   \NC \SubMatch{-1}{-1}     \NC\AR
    \NC \TypedString{aa}  \NC \SubMatch{0}{2}       \NC \SubMatch{0}{2}
    \NC \SubMatch{-1}{-1} \NC \SubMatch{0}{2}       \NC\AR
    \NC \TypedString{aaa} \NC \SubMatch{0}{3}       \NC \SubMatch{1}{3}
    \NC \SubMatch{-1}{-1} \NC \SubMatch{1}{3}       \NC\AR
    \NC \TypedString{ba}  \NC \SubMatch{0}{0}       \NC \SubMatch{0}{0}
    \NC \SubMatch{-1}{-1} \NC \SubMatch{-1}{-1}     \NC\AR
    \HL
    \stoptable

  The matches on an empty input ($ε$) shows that matching an empty
  string is to be considered longer than not matching at all.  The matches on
  input \TypedString{ba} display the leftmost||longest rule in action.  It would
  be possible to match the longer substring \TypedString{b\REMatch{a}}, but the
  leftmost match (the empty string before \type{b}) must be chosen instead.
  The third and fourth row demonstrate how higher||level subexpressions will
  take priority over lower||level subexpressions.  The fourth row also shows
  how the repeated matching rules applies to regular expressions of this kind. 
\stopexample

The rules described above are almost identical to those defined by \POSIX\
\cite[POSIX92].  There are other schemes, but this is definitely the one most
often employed in practice.  However, these rules aren’t ideal for all
situations and have been the subject of some criticism \cite[ClarkeCormack97];
the main complaint being that structured text|<|such as \XML\ \cite[XML]|>|is
often not best subjected to leftmost||longest matching.

\placeintermezzo
  {Why leftmost||longest matching doesn’t agree with \XML.}
  \startframedtext
    This is an intermezzo for anyone familiar with \XML\ and who was thinking
    something along the lines of “why the \dots\ shouldn’t I be using
    leftmost||longest matching on \XML\ input?”  The answer is simple.
    Consider the following \XML\ document:
\startXML
<p>With a little <em>emphasis</em>, a statement
can be made much <em>clearer</em>.</p>
\stopXML
  Now say that we would like to write a regular expression that extracts the
  submatch address of the first \type{<em/>}.  The initial response by many
  beginning regular||expressionists is to write it as \TypedRegex{<em>.*</em>}.
  That’s simple and straightforward~--~“matches everything between the two
  \type{em}||delimiters”.  The only problem is that it doesn’t work as
  expected.  At least not when using leftmost||longest matching.  The regular
  expression will match everything from the first \type{<em>} to the last
  \type{</em>}: \TypedString{<p>With a little \REMatch{<em>emphasis</em>, a
  statement can be made much <em>clearer</em>}.\crlf</p>}.  Clearly not what we
  want.  There are ways to get around this issue|<|by changing the \type{.}
  into something less greedy, such as \type{[^<]}|>|but it would be more
  appropriate if we could simply tell the machinery not to apply the
  leftmost||longest rule in every case.  \insection[greedy and non-greedy
  matching] introduces a way to do just this.
  \stopframedtext

In conclusion, remember that submatch addressing, as described here, does
{\em not} alter any of the properties of the regular expressions we’re dealing
with, so we retain all the wonderful features of regular expressions, while
making them easier to use for practical applications.


\section
  [greedy and non-greedy matching]
  {Greedy and Non||greedy Matching}

To solve the problem inherent in the definition of the \POSIX\ matching rules
described at the end of the previous section, the notion of \Term{greedy} and
\Term{non||greedy} matching was thought of.  Generally, a quantifier, such as
\TypedRegex{*} (Kleene Star) matches the leftmost||longest instance of whatever
it quantifies.  There are, however, occasions where you would prefer that it
matched the leftmost||{\em shortest} instance of whatever it quantifies.
Therefore, the quantifier||modifier \TypedRegex{?} was added to \Perl\ and,
later, many other tools that deal with regular expressions and structured text.
\intable[table:lazy quantifiers] lists the
\index{quantifier+non||greedy}\Term{non||greedy}|<|sometimes referred to as
\index{quantifier+lazy}\Term{lazy}|>|quantifiers that \Perl\ has at its
disposal.  They match as few concatenations of whatever they contain, while
making sure that an overall match will be made.

\placetable
  [here]
  [table:lazy quantifiers]
  {Non||greedy quantifiers in \Perl}
  \starttable[|l|l|]
  \HL
  \NC \bf Quantifier
  \NC \bf Effect on $r$, the quantified regular expression \NC\AR
  \HL
  \NC \type{*?} \NC Matches as few concatenations of $r$ as possible \NC\AR
  \NC \type{+?} \NC Matches as few concatenations of $r$ as possible, but at
                    least one \NC\AR
  \NC \type{??} \NC Matches zero or one concatenations of $r$, as few as
                    possible \NC\AR
  \NC \tttf\leftargument$n$,$m$\rightargument\type{?} \NC
      Matches $n$ to $m$ concatenations of $r$, as few as possible \NC\AR
  \HL
  \stoptable

\indent\index{quantifier+greedy}\Term{Greedy} quantifiers are exactly those
that we have already defined (\TypedRegex{*}, \TypedRegex{+}, \TypedRegex{?},
\TypedRegex-{-$n$\TypedRegex{,}$m$\TypedRegex-}-) and require no additional
syntax or semantics.

\stopcomponent
