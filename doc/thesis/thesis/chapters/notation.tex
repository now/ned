\startcomponent thesis/chapters/notation

\project masters-project
\product thesis

\title{Notation and Typographical Conventions}


This manuscript uses a rather large number of different notations and
typographical conventions that intend to make reading it just a little bit
easier than it would have been without them.  As these conventions are
numerous, we’ll list them all here for easy reference.  So, if you find
yourself lost for the proper meaning of a notation or convention, do refer to
this collection of them.


\subject{Mathematics}

\notation{$\{a_1, a_2, \dots\}$} Set.  The unordered set of
  \DefineTerm{elements} $a_1$,~$a_2$, \dots.

\notation{$(a_1, a_2, \dots, a_n)$} Tuple.  The ordered $n$||tuple of elements
  $a_1$,~$a_2$, \dots,~$a_n$.

\notation{$\naturalnumbers$} Natural.  The infinite set of natural numbers:
  $\{0, 1, 2, 3, \dots\}$ or $\{1, 2, 3, 4, \dots\}$, whichever is more
  appropriate for the occasion.

\notation{$\integers$} Zahl.  The infinite set of integers, i.e.,
  $\{\dots, -2, -1, 0, 1, 2, \dots\}$.

\notation{$x ∈ A$} Set membership.  Element $x$ is a member of set $A$.

\notation{$∉$, $≠$, $≮$, $≯$, \dots}  Negation.  A stricken relation is the
  inverse to the non||stricken relation.  For example, $1⋅2 = 2$, but $2⋅2 ≠ 5$.

\notation{$i$, $j$, $k$, $m$, and $n$} Integer variables.  The variables $i$, $j$,
  $k$, $m$, and $n$ denote integers, i.e., $i, j, k, m, n ∈ \integers$.

\notation{$i_2$, $i_{10}$, $i_{16}$} Number bases.  The integer $i$ expressed in
  binary, decimal, and hexadecimal, respectively.  Binary numbers are expressed
  using 0 and 1; decimal numbers using 0 through 9; hexadecimal using 0 through
  9 and then A through F.

\notation{$A$, $B$, and $C$} Set variables.  Arbitrary sets of elements;
  sometimes with indexes, such as $A_1$, $A_2$, $A_3$, \dots.

\startnotation{$A × B$}
  Cartesian product. The set of all pairs of elements of $A$ and $B$:

    $$A × B = \{\,(a, b) \mid a ∈ A ∧ b ∈ B\,\}.$$
\stopnotation

\notation{$f\colon A → B$} Function signature.  Function $f$ maps each
  element of $A$ (known as the \DefineTerm{parameter}) to an element of $B$
  (known as the \DefineTerm{result}).

\notation{$f(x_1, x_2, \dots, x_n)$} Function application. Function
  $f\colon A_1 × A_2 × \dots × A_n → B$ applied to \DefineTerm{arguments}
    $x_1 ∈ A_1$,~$x_2 ∈ A_2$, \dots,~$x_n ∈ A_n$.

\notation{$P(x)$} Predicate function. A function $P\colon A → \{0, 1\}$
  returning a truth value ($0$~--~false, $1$~--~true).  Such functions are
  often referred to as a boolean function, in honor of the inventor of “the
  calculus of logic” or “boolean algebra”, George Boole \cite[Boole47].

\notation{$\{\,x \mid P(x)\,\}$} Set construction.  The set of all $x$ such
  that $P(x) = 1$.

\startnotation{\Ordo{f(n)}}
  Ordo function.  The set of functions such that

    $$\Ordo{f(n)} = \{\,g(n) \mid ∃ c, n_0(∀ n ≥ n_0(0 ≤ g(n) ≤ cf(n)))\,\}.$$
\stopnotation

\notation{$\lg x$} Base||$2$ Logarithm.  The logarithm with base $2$ of $x$
($\lg x = \log_2 x$).

\startnotation{$A ∪ B$}
  Set union.  The union of sets $A$ and $B$, i.e., the set of elements that
  belong to either $A$ or $B$:
  
    $$A ∪ B = \{\,x \mid x ∈ A ∨ x ∈ B\,\}.$$
\stopnotation

\startnotation{$A ∩ B$}
  Set intersection.  The intersection of sets $A$ and $B$, i.e., the set of
  elements that belong to both $A$ and $B$:
  
    $$A ∩ B = \{\,x \mid x ∈ A ∧ x ∈ B\,\}.$$
\stopnotation

\startnotation{$A \setminus B$}
  Set difference.  The difference between sets $A$ and $B$, i.e., the set of
  elements that belong to $A$ but not $B$:

    $$A \setminus B = \{\,x \mid x ∈ A ∧ x ∉ B\,\}.$$
\stopnotation

\notation{$⋃_{i ∈ B} A_i$} Set union (iteration).  The union
  of sets $A_i$, for all elements $i$ in set $B$; $B$ is often referred to as
  an index set.  The condition on $i$ can of course be generalized to any
  predicate $P(i)$, though it most often takes the form of set membership.

% TODO: fix these four
\notation{$A \subseteq B$} Subset.  Set $A$ is a subset of $B$, i.e., for all
  $x ∈ A$, $x ∈ B$.

\notation{$A \subsetneq B$} Proper subset.  Set $A$ is a proper subset of
  $B$, i.e., $A \subseteq B$ but $A ≠ B$.

\notation{$A \supseteq B$} Superset.  Set $A$ is a superset of $B$, i.e.,
  for all $x ∈ B$, $x ∈ A$.

\notation{$A \supsetneq B$} Proper superset.  Set $A$ is a proper superset
  of $B$, i.e., $A \supseteq B$ but $A ≠ B$.

\notation{\PowersetOf{A}} Power set.  The power set of $A$, i.e., the set of
  all possible subsets of $A$.

\notation{$\sum_{i=1}^n a_i$} Summation.  The sum $a_1+a_2+⋯+a_n$.

\notation{$\prod_{i=1}^n a_i$} Product.  The product $a_1a_2⋯a_n$.

\notation{$n!$} Factorial.  The product $\prod_{k=1}^n k$, for $n>0$, and $1$
for $n=0$.  The factorial of $n$, $n!$, is the number of ways to
arrange $n$ distinct objects.  Each way in which they may be arranged is known
as a \Term{permutation} of the objects.

\startnotation{$\tbinom{k}{n}$}
  Binomial Coefficient. The binomial coefficient is the number of ways to
  choose $k$ objects from a set of $n$ elements ($k≤n$) and is given by
  the equation
  
    $$\tbinom{k}{n} = \frac{k!}{(k - n)!n!}.$$
\stopnotation

Large numbers in text are separated by thin space and a dot is used for the
decimal point, e.g., “\digits{1,000,000} is a relatively small number nowadays,
whatever the topic, yet the number \digits{3.141,592,654}\dots\ is still of
great interest, even to people who are a bit picky with what they eat.”


\subject{Languages, Regular Expressions, and User Input}

Below you’ll find the notation used when discussing languages, regular
expressions, and user input.

\notation{$L$} Language variable.  The variable $L$ denotes an arbitrary
  language, sometimes with indexes: $L_1$,~$L_2$, \dots.

\notation{$L_1∘L_2$ or $L_1L_2$} Concatenation/Juxtaposition.  The
  concatenation of languages $L_1$ and $L_2$.

\notation{$L^∗$} Closure.  The closure of language $L$.

\notation{$L^+$} Positive closure.  The positive closure of language $L$.

\notation{$r$} Regular||expression variable.  The variable $r$ denotes an
  arbitrary regular expression, sometimes with indexes: $r_1$,~$r_2$, \dots.

\notation{$L(r)$} Language.  The language described by regular expression $r$.

\notation{$r_1∘r_2$ or $r_1r_2$} Concatenation/Juxtaposition.  The
  concatenation of the regular expressions $r_1$ and $r_2$.

\notation{$r_1 \Alt r_2$} Union.  The union of regular expressions $r_1$ and
  $r_2$.

\notation{$r^∗$} Closure. The closure of the regular expression $r$.

\notation{$r^+$} Positive closure.  The positive closure of regular expression
  $r$.

When discussed from a mathematical standpoint, the symbols comprising regular
expressions are rendered no differently than other mathematical symbols.
Otherwise, when a regular expression is typed by the user|<|or when not
discussed from a strictly mathematical standpoint|>|at a computer terminal it
is typeset \TypedRegex{like this}.

Input typed by the user is typeset \TypedString{like this} and the match of a
regular expression within its input is typeset \REMatch{like this}.

Computer terminal sessions are typed in a \type{fixed-width typeface} and have
a distinctive background:

\startshellsession
% ls \
> -la
\stopshellsession

Commands are typed at the “\type{% }” prompt.  Long commands are
broken into multiple lines by ending each line with a \type{\} and continuation
lines begin with the string “\type{> }”, which isn’t part of the input.  

While we’re at it, commands that may be run at a computer terminal prompt are
displayed \Command{likethis} in running text.  This notation is borrowed from
the \UNIX\ system manual pages, where commands are described in the first
section of the manual.  This means that the command \Command{man} belongs to
the first section of the manual and its manual pages can be accessed by running
it as follows:

\startshellsession
% man 1 man
\stopshellsession

Finally, whenever a user types input into a text editor or other pieces of
software, the sequence of keystrokes on the keyboard is rendered
\KbdType{l i k e}\KbdKey{Space}\KbdType{t h i s}.  Note how special characters
like \KbdKey{Space}, \KbdKey{Ctrl}, and \KbdKey{Tab} also are rendered
specially.


\subject{Finite Automatons}

Some notational conventions are also used when discussing finite automata.

\notation{$⟨s, w⟩$} Configuration.  A pair describing the state of a finite
  automaton.  The variable $s$ is a state and $w$ a string. 

\notation{$⟨s, u, w, v⟩$}  Tagged Configuration.  A quadruple describing the
  state of a tagged finite automaton.  The variable $s$ is a state, $u$ is the
  part of the input read so far, $w$ is the part of the input that remains, and
  $v$ is a tag||value function.

\notation{$\Yields$} Yields.  Binary relation between two configurations of a
  finite automaton.

\notation{$\RTCYields$} Reflexive Transitive Closure of Yields.  The
  reflexive transitive closure of the \Term{yields} relation between
  configurations of a finite automaton.

\notation{$\UnYields$} Unambiguously Yields.  Binary relation between
  configurations of a tagged finite automaton.

\notation{$\RTCUnYields$} Reflexive Transitive Closure of Unambiguously Yields.
  The reflexive transitive closure of the \Term{unambiguously yields} relation
  between configurations of a tagged finite automaton.

When representing finite automata as transitions diagrams, transitions and
states of particular interest are \color[activetrans]{highlighted}.


\subject{Character Symbols}

\index{character symbol}Character symbols will be important to us in this
manuscript.

\notation{\CharacterName{black star}} Character Name.  The name of a character
  symbol as defined by \Unicode\ or \ASCII\ depending on context.

\notation{\CodePoint{2605}} Code Point.  The position of a character symbol in
  the \Unicode\ character set.

(The character symbol described in the two examples above is $\bigstar$.)


\subject{\BNF\ Grammars}

\index{grammar}\index[BNF grammar]{\BNF\ grammar}\BNF\ is a metasyntax for
describing context||free grammars.  It sees great usage when discussing
programming languages and communication protocols.  The \BNF\ syntax was
originally named after \Name{John}{Backus} (Backus Normal Form) and later, due
to \Name{Donald}{Knuth}’s influence \cite[Knuth64], also \Name{Peter}{Naur}
(Backus||Naur Form) awarding his contributions to the
\index[Algol]{\Algol}\Algol\ 60 report \cite[Naur63, AhoSethiUllman87].

A \BNF\ grammar|<|as typeset in this document|>|will have the general form of
the \BNF\ grammar below describing the way integers are formed in a base ten
system using western digits:

\startbnfgrammar[]
<number>: <digit> | <digit> <number>
<digit>: "0" | "1" | "2" | "3" | "4" | "5" | "6" | "7" | "8" | "9"
\stopbnfgrammar

A rough translation to English of the above grammar is:

\startblockquote
  A \<number> consists of|/|is ($\to$) either a
  \<digit> or ($\,|\,$) a \<digit> followed by another \<number>---which is at
  least another digit long.  A \<digit> consists of any of the symbols
  \type{0}, \type{1}, \dots, \type{9}.
\stopblockquote

Formally, \<number> and \<digit> are known as
\index{grammar+nonterminal}\Term{nonterminals} and \type{0}, \type{1},
through \type{9} as \index{grammar+terminals}\Term{terminals}; a vertical bar
($\,|\,$) indicates \index{grammar+choice in}\Term{choice}.


\subject{Algorithms}

\index{algorithm}Algorithms are listed using \index{pseudocode}pseudocode.
The input and output to the algorithm are listed above the algorithm itself,
which uses a notation reminiscent of many of the programming languages that
descend from \index{\Algol}\Algol\ \cite[Naur63], such as \index{Ruby}Ruby
\cite[Thomas04]\ or \index[C]{\CLang}\CLang\ \cite[KernighanRitchie88].

\inalgorithm[algorithm:euclides]\ displays the Euclidian algorithm that finds
the greatest common divisor of two integers.

\placealgorithm
  []
  [algorithm:euclides]
  {The Euclidian Algorithm.}
  {\startalgorithmio
     \sym{Input:} Two integers $m$ and $n$.
     \sym{Output:} The greatest common divisor (gcd) of $m$ and $n$.
   \stopalgorithmio
\startPSEUDO
while $n ≠ 0$ do
  $r ← m \bmod n, m ← n, n ← r$
return $|m|$
\stopPSEUDO
}

The arrow “←” used in the algorithm above is what we’ll refer to as the
\DefineTerm{assignment} operator.  The operation “$m ← n$” means that the value
of variable $m$ is to be replaced by the value of variable $n$.  The arrow “↔”
is a shorthand for swapping the values of two variables $m$, $n$:
$m ↔ n \equiv t ← m, m ← n, n ← t$, where $t$ is an unbound|<|new|>|variable
acting as a temporary intermidiary between the two variables’ values.

We will also be using some data structures when writing our algorithms:

\notation{$Q \Push x$} Push.  Push element $x$ onto \index{queue}queue or
  \index{stack}stack $Q$.  If $Q$ is a queue, append $x$ to the end of $Q$;
  otherwise, push $x$ onto the top of $Q$.

\notation{$x \Pop Q$} Pop.  Pop element $x$ off of queue or stack $Q$.  If $Q$
  is a queue, set $x$ to the first element of $Q$; otherwise, set $x$ to the
  topmost element of $Q$.  Then remove the chosen element.


\subject{Flowcharts}

\index{flowchart}Flowcharts are used to give a visual overview of the flow of
information in a system.  In this manuscript, they are mainly used as a visual
representation of algorithms.

A flowchart is a set of shapes interconnected by arrows.  The shapes represent
entities in the system, and arrows represent the interconnections between them.

The entities we’ll be using are displayed in
\infigure[figure:flowchart entities].  As an example of a flowchart,
\infigure[figure:digestive system]\ displays a simplification of the human
digestive system.

\placefigure
  [here]
  [figure:flowchart entities]
  {Flowchart entities used in this manuscript.}
  {\FLOWchart[flowchart entities]}


\placefigure
  [here]
  [figure:digestive system]
  {A simplified representation of the human digestive system as a flowchart.}
  {\FLOWchart[digestive system]}


\subject{Miscellany}

{\URL}s are typeset in a fixed||width typeface that looks \from[likethis]\ and
are “clickable” in the interactive screen version (\PDF\ \cite[PDF03]) of
this manuscript. Thus, if you click it with the left mouse button, you will be
taken to the document to which the \URL\ refers to by your web browser;
hopefully not Microsoft Internet Explorer which even \CERT\ and Microsoft
itself has deemed unusable \cite[Krebs04, Boutin04]---there, the first, and
hopefully last, semi||political statement in this manuscript.

\Note Paragraphs beginning with a

  \midaligned{\reuseMPgraphic{note}}

in the margin cover advanced topics and require a heightened level of
concentration to be fully understood.  You may even find that glancing over
them on the first reading works to your benefit, as their contents is not
always required reading.  Of course, the question is if you should have read
this paragraph in the first place, seeing as there is an admonition in the left
margin of this paragraph.  If you had not, you would not have known that you
should not have, so you should have, after all\dots


\subject{Notes on the Form of this Manuscript}

I admit it: I {\em love} footnotes.  I guess I’ve read a couple of Terry
Pratchett books too many; as \Name{Eric S.}{Raymond} points out in
\cite[Raymond03], “[Pratchett’s] use of footnotes is quite\dots inspiring”.
Therefore, you will see the odd footnote on seemingly irrelevant topics.  I beg
you to at least give them a glance, as they are meant to provide some sort of
entertainment along the way.

Many of the rules that have governed the contents and form of this manuscript
are due to the influence of \PublicationTitle{The Chicago Manual of Style}
\cite[Chicago03]\ and \PublicationTitle{The Elements of Style} \cite[Strunk99].
These are truly wonderful guides in the jungle that is writing and typesetting.

Other influences on the writing of this manuscript were due to
\Name{Donald}{Knuth}’s \PublicationTitle{Mathematical Writing} \cite[Knuth89],
\Name{Bill}{Walsh}'s \PublicationTitle{Lapsing Into a Comma} \cite[Walsh00],
and \PublicationTitle{Effective Writing} by \Name{Bruce}{Ross-Larson}
\cite[RossLarson99].


\subject{Notes on the Typesetting of this Manuscript}

This manuscript was typeset using the \ConTeXt\ macro package \cite[Hagen01]\ 
for \TeX\ \cite[Knuth86]\ using the \PDFTEX\ engine \cite[Thanh00,Thanh05].
\ConTeXt\ was written by \Name{Hans}{Hagen} et al. and is distributed by
\cap{pragma} Advanced Document Engineering (\PRAGMA), \TeX\ by
\Name{Donald E.}{Knuth}, and \PDFTEX\ by \THANH.  Additional macros were
provided by \Name{Taco}{Hoekwater} (the \BIBTEX\ bibliography module for
\ConTeXt), \Name{Giuseppe}{Bilotta}|/|\Name{Michal}{Marvan} (the natural math
module \Nath), and the author himself.

Figures were rendered using \METAPOST\ \cite[Hobby92]\ and \METAFUN\
\cite[Hagen02], the graphic extensions to the typeface renderer \METAFONT\ by
\Name{Donald E.}{Knuth} \cite[Knuth86b], written by \Name{John D.}{Hobby} and
\Name{Hans}{Hagen}, respectively.  Additional macros for rendering finite
automata and trees were written by the author himself.

This manuscript has been typeset in the Computer Modern typeface family, also
due to \Name{Donald E.}{Knuth} \cite[Knuth86c].

The cover displays one of the longest|<|if not {\em the} longest|>|regular
expressions ever created.  It can be used to validate email addresses that
conform to the definition found in \RFC\ 822 \cite[RFC822]\ and is almost 6500
symbols long.

\stopcomponent
