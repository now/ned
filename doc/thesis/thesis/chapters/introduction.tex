\startcomponent thesis/chapters/introduction

\project masters-project
\product thesis

\chapter{An Introduction}

\Term{Pattern matching} has a broad area of application in computer science.
It is used in database query systems, data processors, lexical analyzers, and
text editors.  Of special interest to us is its use in the last.

Text editors are one of the main tools for software development, since sooner
or later you, as a software developer, will inevitably be required to alter the
contents of your source code.  The usefulness of an editor varies greatly
depending on what kinds of features it boasts, and even though opinions on
which text editor is to prefer over all others, an editor would not be of much
use unless it provides a good way of searching the contents of the file
currently being edited.  This feature existed in the first on||line text
editors\footnote{On-line in the sense of being connected to a terminal,
providing output on a television||like terminal screen.} and the usefulness of
\Term{regular expressions} was recognized very early as the means of specifying
the patterns used as search terms.

In fact, it can be argued that the introduction of regular expressions to {\em
real} computers was a result of the need for a way to represent search patterns
in the \Qed\ \cite[RitchieThompson70,RitchieQEDHistory]\ editor.
\Name{Ken}{Thompson} devised a method that turned a regular expression into
machine code\footnote{This was the first use of on||the||fly
compiling\footnote{Another term for this is
just||in||time, or JIT, compiling.}, i.e., the
process of generating computer instructions and running them as the situation
demanded, in history \cite[KernighanPike99].}
\cite[Thompson68,HopcroftMotwaniUllman01]\ that executed a nondeterministic
finite automaton representation of the original regular expression.  This
proved very fruitful and spawned a lot of research interest into the subject of
regular expressions and finite automata for use in computers.

This manuscript attempts to survey the necessary theoretical background behind
regular expressions and text editors.  In doing this, we’ll begin by discussing
the subjects of symbols, alphabets, strings, languages, grammars, regular
expressions, finite automata of various types, character sets, code points, and
character encodings in \inchapter[definitions].  In \inchapter[the real world]
we’ll then look at how regular expressions can be used in real||world
applications, and some of the common extensions such use has introduced over
time to the syntax and even semantics of regular expressions.
\inchapter[patterns] will then introduce the problems of pattern matching and
submatch addressing, also discussing how regular expressions may be applied to
solve these problems.  Then, in \inchapter[construction], we’ll look at how we
can create finite automata that we may then simulate the execution of on a
computer.  Being somewhat more light||hearted, \inchapter[text editors -
external] discusses the external functionality of text editors, followed by a
theoretical discussion of the internal functionality of text editors in
\inchapter[text editors - internal].  \inchapter[regexes in text editors] then
looks at how regular expressions have been used in the command sets of two
modern text editors that use them extensively.  The following chapter,
\inchapter[new syntax], introduces a suggestion for a new syntax for regular
expressions that will be useful in expressing regular expressions in an
interactive processes, such as a text editor.  A software library that
implementations these regular expressions using efficient finite automata is
then discussed in \inchapter[implementation
- regexes], followed by the implementation of an example text||editor that
tries to put these regular expressions to good use in its command set.
Finally, \inchapter[future work] discusses further ideas relating to the
subjects introduced in this manuscript that one may want to investigate
further.

\stopcomponent
