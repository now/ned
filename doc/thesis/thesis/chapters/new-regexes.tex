\startcomponent thesis/chapters/new-regexes

\project masters-project
\product thesis

\chapter
  [new syntax]
  {A New Syntax for Regular Expressions}


% TODO: how do we discuss semantics here
This chapter will cover the work that has been done on trying to define and
implement a new regular expression syntax and associated semantics for it.

The new syntax was designed with a couple of goals in mind.  A new regular
expression syntax should:

\startitemize
  \item Be easy to read and understand
  \item Be as succinct as possible
  \item Make simple things easy to express, and hard things possible
\stopitemize

These goals rely on each other in a sense.  Succinctness makes things easy to
read, as there is less of it.  Understandability may suffer from succinctness,
but this is counter||weighted by simple things being easy to express.

As not all people think alike, it’s important to provide means for users to
customize the language themselves.  This is much the same workings that make
colloquializations so important.  Anyhow, we add another point to our design
goals:

\startitemize
  \item Allow the user to extend the syntax
\stopitemize

The next couple of sections will introduce a new syntax that tries to achieve
the goals outlined above.



\section{An Introduction}

We’ll begin by presenting you with the whole grammar all at once as
\inbnfgrammar[bnf:new regex syntax] and then discuss the parts of the grammar
that need clarification.  This grammar is to be known as \NRE, short for New
Regular Expression among other possible expansions\footnote{Can you catch them
all?}, notation, following the naming-scheme of the notations defined by
\POSIX\ (\inchapter[the real world]), i.e., \BRE\ and \ERE.

\placebnfgrammar
  []
  [bnf:new regex syntax]
  {A complete grammar of the new regular expression syntax: \NRE.}
  {\startbnfgrammar[]
     <regex>: <concat> | <regex> "|" <concat>
     <concat>: <piece> | <concat> <piece>
     <piece>: <atom> | <atom> <quantifier>
     <atom>: <assertion>
           | <literal>
           | "."
           | "<[" <symbol range> "]>"
           | '<"' <escaped string> '">'
           | "<'" <string> "'>"
           | "<" <rule> ">"
           | "[" <regex> "]"
           | "(" <regex> ")"
     <quantifier>: <greedy quantifier> | <non-greedy quantifier>
     <greedy quantifier>: "*" | "+" | "?" | <range>
     <non-greedy quantifier>: <greedy quantifier> "?"
     <range>: "<" <integer> ">"
            | "<" <integer> "," ">"
            | "<" <integer> "," <integer> ">"
            | "<" "," <integer> ">"
     <assertion>: "^" | "$" | "^^" | "$$" %$

     <literal>: <non-{\tttf\texescape} symbol> | "\" <special symbol>
     <symbol range>: \dots
     <escaped string>: <escaped symbol> | <escaped string> <escaped symbol>
     <escaped symbol>: <non-{\tttf\texescape} symbol> | "\n" | "\t" | "\" <escaped special symbol>
     <escaped special symbol>: '"' | "\"
     <string>: <non-\type{'} symbol> | <string> <non-\type{'} symbol>
     <rule>: <non-\type{>} symbol> | <rule> <non-\type{>} symbol>
     <integer>: <digit> | <integer> <digit>
     <digit>: A \Unicode\ digit
     <non-{\tttf\texescape} symbol>: $\hbox{\Unicode} \setminus \{ \text{\tttf\texescape} \}$
     <non-\type{'} symbol>: $\hbox{\Unicode} \setminus \{ \text{\tttf\bnfsinglequote} \}$
     <non-\type{>} symbol>: $\hbox{\Unicode} \setminus \{ \text{\type{>}} \}$
     <special symbol>: "|" | "." | "<" | "(" | "*" | "+" | "?" | "\" | "^" | "$" %$

   \stopbnfgrammar}

At first it may seem like a rather complicated grammar; compared to the one
for regular expressions in \Vim, see
\inbnfgrammar[vim regular expression syntax:bnf:vim regex], though, it’s
actually quite simple.  The rules that deal with escaping (\<escaped \dots>
and \<non-\dots>) obfuscate an otherwise rather simple grammar.

% TODO: we haven't discussed properties of code points
Please note that the meaning of \Unicode\ in the grammar is meant to represent
the union of rules for the \Unicode\ character set.  That is, a
\<non-{\tttf\texescape} symbol> is any valid \Unicode\ symbol except a
backslash and a \<digit> is any valid \Unicode\ digit, i.e., any code point
that has the Nd property.

The following sections will introduce the various parts of the grammar, though,
the rules for \<regex>, \<concat>, and \<piece> introduce nothing new and we’ll
not discuss them further.


\section{New Atoms---Clearing the Way}

Our syntax introduces a range of new atoms.  The \<atom> rule has in fact been
the center of our design effort.  


\section{Two Variants of Subexpression Grouping}

One of the problems with many other regular expression syntaxes is that they
have a hard time separating subexpressions that should be considered for
submatch addressing from those that shouldn’t.  It’s often desirable to
perform grouping without the contained subexpression being considered for
submatch addressing, as keeping track of it will induce an execution time
penalty.  Also, if the subexpression is uninteresting, this will increase
recall, which lowers precision, to use terms borrowed from information
retrieval.  Simply put, it makes it harder for the user to keep track of the
interesting subexpressions, if uninteresting ones can't be kept out.

The standard syntax for subexpression grouping is to use either
\TypedRegex{(}\dots\TypedRegex{)} or \TypedRegex{\(}\dots\TypedRegex{\)}
depending on if the syntax is \BRE\ or \ERE.  Such subexpressions are always
considered for submatch addressing; neither grammar, as defined by
\POSIX, provide means for the user to disable submatch addressing.  \Perl\
solves this by extending the syntax with the following constructions:

\startbnfgrammar[]
  <addressed subexpression>: "(" <regex> ")"
  <unaddressed subexpression>: "(?:" <regex> ")"
\stopbnfgrammar

This works fine, but the syntax is very cumbersome.  Having to type two extra
symbols (\type{?:}) to turn off submatch addressing seems unnecessary.  It
turns out that this is actually the case.

The solution in our grammar is to use two different delimiters, one set for
addressed subexpressions and one for unaddressed ones:

\startbnfgrammar[]
  <addressed subexpression>: "(" <regex> ")"
  <unaddressed subexpression>: "[" <regex> "]"
\stopbnfgrammar

The use of square brackets (\type{[} and \type{]}) would cause problems with
the \BRE\ and \ERE\ syntaxes, as they are used for character ranges.  Our
grammar, however, uses a different syntax for character ranges,
\TypedRegex{<[}\dots\TypedRegex{]>}, which solves this problem.  Some may now
argue that we have made unaddressed subexpressions easier to type at the cost
of making character ranges harder.  While this is true, we’ve done so with good
reason: character ranges are a thing of the past.  Character ranges work fine
for a highly restricted character set, such as \ASCII, where it's clear what a
range such as \TypedRegex{[a-z]} means, but crumble under the weight of a
larger one, such as \Unicode, where the question of whether the range should
include lowercase characters other than those from the English language or not
arises.  Our solution is to make character ranges possible to specify, but
users are provided with a much more versatile tool, namely the \<rule>.


\section{The Rules That Dictate Our Regular Expressions}

It’s now time to introduce the most innovative part of the grammar, the
\<rule>.  A \<rule> is a name of another regular expression that is to be
substituted into the regular expression.  An example is if we have previously
defined a regular expression \TypedRegex{<digit>} as

\placeregex
  {none}
  \TypedRegex{<digit> = 1|2|3|4|5|6|7|8|9|0}

and would like to define a regular expression \TypedRegex{<integer>} that
matches a concatenation of one or more digits.  We might do this as

\placeregex
  {none}
  \TypedRegex{<integer> = <digit>+}

And we can then, in turn, use \TypedRegex{<integer>} in another regular
expression.

This may seem to be nothing more than a macro preprocessor, and that is
almost true.  The strange thing is, this has never been done before, at least
not for interactively used regular expression syntaxes.  Both the \FSA\
Utilities \cite[vanNoord00]\ and Xerox finite-state software \cite[Beesley03]\
have ways of naming regular expressions in this way.  The difference is that
both are meant for non||interactive use.  The same goes for \Perl\ 6's regular
expressions that will provide a syntax for naming regular expressions similar
to that presented above \cite[Wall02,Randal02,Conway02].  Their exact semantics
are still unclear, though, as this part of \Perl\ 6 has yet to be written.

It should be noted that the macro preprocessor is actually not a preprocessor
at all, but built into the regular expression parser.  Thus, a \<rule> is
expanded not as the contents of the rule, but as a unit, or \<atom> in this
case.  This saves us from having to wrap the definition of \TypedRegex{<digit>}
in square brackets, for example.

\startexample[example:floating point number]
  Let’s write the regular expression found in
  \in[the real world:regex:posix floating point number]\ using our new syntax.

  Perhaps the easiest way to go about this task is to break it up into
  components.  We begin by creating a named regular expression for the optional
  sign that appears twice in the regular expression:

  \placeregex
    {none}
    \TypedRegex{<sign> = [+|-]?}

  Simple enough.  Next, let’s define a regular expression that matches a float
  without the optional sign and exponentiation:

  \placeregex
    {none}
    \TypedRegex{<float> = <digit>*[<digit>\.|\.<digit>]<digit>*}

  This is a useful regular expression in its own right, as it will match many
  floating point values.  We do, however, want to define a regular expression
  that matches a wider range of values and we begin with one that may begin
  with an optional sign:

  \placeregex
    {none}
    \TypedRegex{<sfloat> = <sign><float>}

  Finally, we write a regular expression that matches an optionally signed
  floating point value with an optional exponentiation:

  \placeregex
    {none}
    \TypedRegex{<sfloate> = <sfloat>[[e|E]<sign><digits>]?}

  The \TypedRegex{<sfloate>} regular expression can then be used whenever we
  wish to match an optionally signed, optionally exponentiated floating point
  value.
\stopexample

The previous example shows a lot of the power of rules.  It allows us to give
meaningful names to regular expressions that we may want to reuse.  We may, for
example, use \TypedRegex{<sfloate>} as a regular expression, instead of typing
the whole regular expression of
\in[the real world:regex:posix floating point number].  It also makes it easier
to change the definition of the regular expression, as, if we reuse it
properly, all instances of it will follow suit.

Some things remain to be studied, though.  As the example of floating point
values showed, we create a couple of regular expressions along the way to our
final one.  Some may be useful, others not.  Thus, there is a question of how
to hide regular expressions that aren't meant to be used beyond a certain set
of regular expressions.  This is much the same problem as with hiding private
methods in object||oriented programming or similar.

Some sort of namespacing may also be desirable, so that a set of related
regular expressions can be grouped together as a single unit.  For interactive
use, this would make it possible for users to choose a set of regular
expressions that they would like to use depending on need.

These questions are open for discussion and will receive more attention in the
future.  For now, the important thing is that they allow us to experiment with
naming regular expressions and to see what kind of reuse we can expect from
them.


\section{Literal Strings---Escaped and Otherwise}

A problem with many regular expression syntaxes is that they use far too many
operator symbols.  This forces a lot of escaping to be used which makes the
regular expression hard to read.

While our syntax doesn't suffer as much from this problem as many other, it’s
still an issue that needs some attention.  The solution is to allow sections of
the regular expression to be left uninterpreted.  We add two constructions to
the \<atom> nonterminal:

\startbnfgrammar[]
  <atom>: '<"' <escaped string> '">' | "<'" <string> "'>"
\stopbnfgrammar

Both constructions allow the user to write an uninterpreted string in the
middle of a regular expression.  The difference between the two is slight:  The
first allows certain symbols to appear, such as a \type{"}, which would
otherwise be seen as part of the terminating \type{">} and a
\CharacterName{line feed (fd)} \type{\n} or \CharacterName{character
tabulation} \type{\t};  the second doesn't provide an escaping method, and thus
isn’t interpreted in any way.  This means that an apostrophe or single quote
\type{'} may not appear within the string.

An example of the usefulness of this syntax is in creating a regular expression
for matching certain regular expressions:

\placeregex
  {none}
  \TypedRegex{<"<sfloat>[[e|E]<sign><digits>]?">}

This regular expression will match the regular expression we wrote in
\inexample[example:floating point number] itself.


\section{Quantifiers and Ranges}

% TODO: shuld be using \TypedRegex here
There are several ways of expressing repetition in our syntax.  It contains the
standard ones, such as \type{*}, \type{+}, and \type{?}, but also a new syntax
for the bounded repetition usually expressed using braces, see
\insection[the real world:iteration shorthands].  The operators are described in
\intable[table:repetition operators].

\placetable
  []
  [table:repetition operators]
  {The repetition|/|iteration|/|quantifier|/|closure operators of our syntax.}
  \starttable[|l|lp(25em)|]
  \HL
  \NC \bf Operator                    \NC \bf Repetitions of the quantified regular expression  \NC\AR
  \HL
  \NC \type{*}                        \NC zero or more                                          \NC\AR
  \NC \type{+}                        \NC one or more                                           \NC\AR
  \NC \type{?}                        \NC zero or one                                           \NC\AR
  \NC \type{<}$n$\type{>}             \NC exactly $n$                                           \NC\AR
  \NC \type{<}$n$\type{,>}            \NC $n$ or more                                           \NC\AR
  \NC \type{<}$n$\type{,}$m$\type{>}  \NC $n$ to $m$                                            \NC\AR
  \NC \type{<,}$m$\type{>}            \NC $0$ to $m$                                            \NC\AR
  \HL
  \stoptable

The advantage of using pointed braces over curly ones is that we use one less
symbol in our syntax\footnote{A second advantage is that they are a lot easier
to write in a \TeX\ document than their curly counterparts.}.

There are also non||greedy iterations, that are expressed in the same way as
introduced in \insection[patterns:greedy and non-greedy matching], by adding
the \TypedRegex{?}||suffix to the iterator operator.


\section{Assertions}

There isn't much fun going on for assertions in our syntax.  Currently, the
only ones supported are those that match at the beginning and end of a line,
\TypedRegex{^} and \TypedRegex{$}, and those that match at the beginning and
end of the input, \TypedRegex{^^} and \TypedRegex{$$}.  This is such, as the
first two have traditionally been used for the line||wise matching.  However,
as we are trying to break free from the line||wise paradigm (see
\insection[regexes in text editors:problems]), the two pairs may swap meanings
in a future version of the syntax, depending on what turns out to work best.


\section{Unicode and Symbol Ranges}

Of great importance to our syntax|<|and our implementation of it|>|was to
use \Unicode\ as the character set.  In doing so, we allow any text
written in a human language to be described by a regular expression using our
syntax.  This was important, as we realize that not all input will be written
using the limited \ASCII\ character set, for example.

The exact syntax of symbol ranges remains to be defined.  It’s clear, however,
that only allowing the simple kind of ranges that are described in
\insection[the real world:character ranges]\ are not sufficient in describing
the kind of languages that we want to match, seeing as how we use \Unicode\ as
our character set.

\Unicode\ Technical Standard \#18 \cite[Davis04]\ discusses how to use
\Unicode\ together with regular expressions.  This document is, however, far
too concerned with the exact syntax to use, which is greatly influenced by the
syntax used in \Perl\ 5.  It’s therefore not quite the answer to our question
as to how to merge regular expressions with the vast expanse that is the
\Unicode\ character set.

\section{Some Tasteful Examples}

Before we end this chapter and go on to discuss the implementation of this new
regular||expression grammar, let’s step through some more examples of how these
regular expressions can be used together with an editor that possesses a
command||set similar in nature to that of the text editor \SAM.

\startexample
  In \cite[Pike87], Rob Pike suggests, through step||wise refinement, a
  method for renaming a variable in a \CLang\ source||file.  We’ll now see how
  the regular expressions used in this method may be simplified using {\NRE}s
  and how this simplifies the method as a whole.

  The problem is the following: we want to rename the variable \C{n} to \C{num}
  throughout the buffer.

  Our first attempt at a solution is to extract every \C{n} and change it to
  \C{num}:

    \starteditorinput
      \type{, x/n/ c/num/}
    \stopeditorinput

  This solution isn’t all that great, as it will change {\em every} ‘n’ found
  in the buffer, not just variables that are named thusly.  A better solution
  is to extract all variables in the buffer and, depending on if they are our
  desired variable \C{n}, change them:

    \starteditorinput
      \type{, x/[A-Za-z_][A-Za-z_0-9]*/ g/n/ v/../ c/num/}
    \stopeditorinput

and, using {\NRE}s:

    \starteditorinput
      \type{, x/<ident>/ g/^^n$$/ c/num/}
    \stopeditorinput

  The pattern \TypedRegex{[A-Za-z_][A-Za-z_0-9]*} matches \CLang\ identifiers.
  We haven’t defined the \TypedRegex{<ident>} rule expands to, but one way that
  works is:

    \placeregex
      {none}
      {\TypedRegex{<identhead> = [A-Za-z_]} \crlf
       \TypedRegex{<identtail> = [A-Za-z_0-9]*} \crlf
       \TypedRegex{<ident> = <identhead><identtail>}}

  We of course don’t have to break the definition up into three pieces, but it
  makes it somewhat clearer what’s going on.  We could also choose freely
  whether to put the \TypedRegex{*} in the definition of
  \TypedRegex{<identtail>} or in \TypedRegex{<ident>}.  The earlier was chosen,
  as it seems more natural, at least to me.

  Returning to our commands, in \SAM\ we are forced to use two guards,
  \type{g/n/} and \type{v/../} before we change the variable’s name.  The first
  checks that the variable’s name contains an ‘n’ and the second makes sure
  that the variable’s name doesn’t contain two (or more) characters.  Using
  {\NRE}s, we can write this guard in one go as \type{g/^^n$$/}.  The
  surrounding assertions will make sure that the matched variable is \C{n} as
  we wished.

  Another refinement of our command is to not alter ‘n’s found inside character
  constants and string literals.  We do this by using the \type{y} command:

    \starteditorinput
      \type{, y/'\\?.'/ y/"[^"\\]*(\\.[^"\\]*)*"/} \crlf
      \type{  x/[A-Za-z_][A-Za-z_0-9]*/ g/n/ v/../ c/num/}
    \stopeditorinput

and, using {\NRE}s:

    \starteditorinput
      \type{, y/<chconst>|<string>/ x/<ident>/ g/^^n$$/ c/num/}
    \stopeditorinput

  Again, we use rules to simplify our commands.  The rules can be defined as:

    \placeregex
      {none}
      {\TypedRegex{<chconst> = '\\?.'} \crlf
       \TypedRegex{<string> = "[^"\\]*(\\.[^"\\]*)*"}}

  The regular expression for matching strings is particularly complex and uses
  some tricks to speed up matching.  As this pattern matches strings as defined
  in many programming languages, defining it only once saves us a lot of memory
  (human memory that is) and typing.

  Again returning to our commands, the use of {\NRE}s makes the command a lot
  shorter, thus easier to type and read, but also a lot easier to understand.
  It’s much easier to see what the intent of the command is when it isn’t
  cluttered with the details of the patterns.  Instead, we can read it aloud to
  understand its meaning: “In the whole buffer (\type{,}), look for identifiers
  (\type{x/<ident>})  between matches of character constants and string
  literals (\type{x/<chconst>|<string>/}) that consist of the sole character
  ‘n’ (\type{g/^^n$$/}) and change them to “num” (\type{c/num}).  Such a
  translation isn’t as easy to make for the first command.
\stopexample

\stopcomponent
