\startcomponent thesis/chapters/text-editors-external

\project masters-project
\product thesis

\chapter
  [text editors - external]
  {Text Editors: External Functionality}


So far we’ve devoted all our typesetting to languages, grammars, and, in
particular, regular expressions and finite automata.  It’s time we begin
discussing the setting in which the knowledge we’ve gained will be useful,
namely {\em text editors}.

“Eh, what's a text editor, then?”, we hear our hypothetical reader
ask.  Well, in a broad sense, any piece of computer software that allows its
user to create and alter a string can be said to be a text editor; though, we
would be bending the meaning of \Term{text} a bit.  Most commonly, however, we
consider certain software to be text editors, others word processors or musical
score editors.  The difference between a text editor and a word processor is
the most important for they are the two that are most commonly confused.

A \DefineTerm{word processor} is a piece of software that usually comes bundled
with desktop publishing software.  It presents the user with a visual
representation of how the document being edited will look upon printing while
it’s still being altered by said user.  This editing methodology is commonly
referred to as “What You See Is What You Get”, or its better||known acronym --
\WYSIWYG, and has become popular among users otherwise unfamiliar with
computers, as it tends to be easy to use (and misuse, or even abuse).  A word
processors is used for typesetting documents for printouts.  Examples include
OpenOffice.org \cite[OpenOffice]\ and LyX \cite[LyX].

A \DefineTerm{text editor} is a piece of software that usually comes bundled
with the operating system of a computer.  It presents the user with a visual
representation of the actual contents of the document being
edited\footnote{Actually, some won’t even provide the luxury of a visual
representation, requiring its user to issue special commands to display a given
portion of the document.  We don’t need to go into the details.}.  This is
useful if your document doesn’t require advanced typesetting (or if you prefer
to issue the typesetting commands yourself, instead of letting an inferior
piece of computer software do it for you).  Typical examples of such documents
are software source code, which is the specification of a piece of software,
written in a programming language understood by the computer, or a
configuration file of the operating system.  Good examples of text editors are
\ED, \Qed, \VI, \Vim, and \EMACS\ \cite[Stallman81,Stallman02].

In this chapter and the next we will discuss the external and internal workings
of a text editor, which will make it possible for us to reason about the role
of regular expressions in text editors.  We begin with the most basic
functionality and work our way toward more advanced topics as we gain the
necessary knowledge.


\section{Some Necessary Definitions}

To discuss the workings of a text editor, we must first define some of the
new terms we’ll be using.
% TODO: a bit too short

\startdefinition
  A \DefineTerm{file} or \DefineTerm{document} is a finite sequence of symbols,
  found on some storage media, e.g., a hard drive, \CDR, \CDRW, \DVD, magnetic
  tape, or similar.  We only concern ourselves with files whose its contents is
  intelligible to us, e.g., an email, software source code, or a master’s
  thesis.
\stopdefinition

\startdefinition
  A \DefineTerm{buffer} is the text editors representation of a document.  It
  usually consists of the contents of the document plus any alterations that
  have been made to it that are yet to be written to the storage media.  This
  representation is often stored in a computer’s random||access memory, or
  \RAM; however, certain buffering strategies only keep the regions of a file
  currently being viewed and edited in memory.  One often speaks of “editing a
  buffer” when one is editing a file in a text editor, as the text editor
  mostly deals in terms of buffers, not files.  For mathematical purposes, we
  will denote the set of buffers as $\Buffers$.
\stopdefinition

\startdefinition
  Each buffer in a text editor has a notion of the current position where the
  user wants to make changes, known as \DefineTerm{point}.  There’s only one
  point in a given buffer.  In this manuscript we’ll use the word
  “point” as a proper noun (though, without initial capitalization),
  i.e., allowing constructions like “at point”, “of point”, and “move point”.
\stopdefinition

% TODO: do we need to define marks and regions?  Are they interesting?
%\startdefinition
%  A buffer also has a set of \DefineTerm{marks} associated with it.  A mark is
%  a saved point, i.e., a position in the buffer.  Marks are often used for more
%  advanced editing commands|/|operations and to restore point to a previous
%  position.
%\stopdefinition
%
%\startdefinition
%  Finally, a buffer 
%\stopdefinition

With these definitions, we can give a mathematical representation of a buffer
$B$:

  \[B = (w, p),\]

where $w$ in $Σ^∗$ is the text of the buffer and $p$ in $\naturalnumbers$ is
the position of point in the buffer.  Just to make things as clear as some very
clear substance, we also define the term \Term{text}.

\startdefinition
  A \DefineTerm{text} is a finite sequence of symbols and is considered to be a
  term equivalent to the term \Term{string} defined in
  \indefinition[definitions:definition:string].
\stopdefinition


\section{Inserting Symbols}

The single most important functionality of any text editor is to allow its user
to insert symbols into a buffer.  With the \Term{insert operation} we can
create any sequence of symbols.

\startdefinition
  The \DefineTerm{insert operation} of a text editor inserts a symbol into a
  buffer at point.  Mathematically, we may define this as a function that takes
  a buffer and a symbol as arguments and gives a new buffer as a result:

  \[\Finsert\colon \Buffers×Σ → \Buffers, \\
    \Finsert((w, p), b) = (a_1 a_2 \dots a_{p - 2} a_{p - 1} b
                           a_p a_{p + 1} \dots a_{|w|}, p + 1).\]

  Note that we move point forward one symbol so that the next insert
  operation will take place after the symbol that was just inserted.
\stopdefinition

\starttheorem
  Using only the insert operation, any string can be rendered.
\stoptheorem

\Proof  We use construction to create the string $a_1 a_2 \dots a_n$:

  \[(a_1 a_2 \dots a_n, p) =
        \underbrace{\mathstrut\Finsert(\Finsert(\dots\Finsert(}_{n \text{ times}}
          (ε, 1), a_1), a_2), \dots, a_n).\]
\QED

In most text editors, simply typing the symbols you would like to insert on the
keyboard (or other computer input device) will invoke thu \Finsert\ function.
Some text editors, however, require that you enter a special \Term{editing
mode} where this is true.  Such text editors often have many different
\Term{editing modes}, such as the just described \Term{insert mode}, a
\Term{command mode} where the text editor expects the user to type in the
\quotation{names} of commands to be executed (such as the command to change
mode), and perhaps a \Term{replace mode} where symbols are replaced instead of
simply shifted to the right when new symbols are inserted as the definition
above suggests.  We are jumping the gun here, as we will discuss editing modes
in greater detail later on; however, it is important to note that not all text
editors work alike.


\section{Deleting Symbols}

The second most important functionality of any text editor is to allow its user
to delete symbols from a buffer.  With the \Term{insert operation} and
\Term{delete operation} we can transform any sequence of symbols into any other.

\startdefinition
  The \DefineTerm{delete operation} of a text editor deletes a symbol from a
  buffer at point.  Mathematically we define this as

  \startnathequation
    \Fdelete\colon \Buffers → \Buffers, \\
    \Fdelete((w, p)) = \wall
      (a_1 a_2 \dots a_{p - 2} a_{p - 1} a_{p + 1} a_{p + 2} \dots a_{|w|}, p'), \\
      \qquad p' = \max(\min(p, |w|), 1). \return
  \stopnathequation
\stopdefinition

\starttheorem
  Using only the insert and delete operations, any string can be transformed
  into any other.
\stoptheorem

\Proof  We use construction to create the string $a_1 a_2 \dots a_n$:

  \[(a_1 a_2 \dots a_n, p') = \\ \quad
      \underbrace{\mathstrut\Finsert(\Finsert(\dots\Finsert(}_{n \text{ times}} \\ \qquad
      \underbrace{\mathstrut\Fdelete(\Fdelete(\dots\Fdelete(}_{m \text{ times}}
        (b_1 b_2 \dots b_m, p)))), \\ \qquad
      a_1), a_2), \dots, a_n).\]
\QED


\section
  [moving point]
  {Moving Point}

The third most important functionality of any text editor is to allow its user
to alter the position of point.

\startdefinition
  The \DefineTerm{move||to operation} of a text editor moves point to a new
  position in the buffer:

  \startnathequation
    \Fmoveto\colon \Buffers×\naturalnumbers → \Buffers, \\
    \Fmoveto((w, p), q) = (w, q'),\qquad q' = \min(q, |w| + 1).
  \stopnathequation
\stopdefinition

It’s worth noting that most text editors allows point to be moved relative to
its current position, which in most cases works to the users advantage---humans
are better at doing relative positioning than absolute positioning.


\section{Intermission~--~A Text Editing Session}

So far in this chapter, we’ve only discussed low||level operations of text
editors.  It would be a very painstaking task to edit text using only these
operations.  It is therefore time to show some examples of how modern text
editors present their users with the contents of buffers and how they allow
them to navigate and alter their contents.  We’ll use \Vim\ as our editor, as
it provides many nice high||level operations and is almost ubiquitously
available.

\placefigure
  [here]
  [figure:vim startup]
  {\Vim\ just after startup.}
  {\externalfigure[vim startup]}

When you run \Vim\ from a computer terminal (or other) you will be presented
with the display showed in \infigure[figure:vim startup].  The bottom two lines
are for status messages and advanced command input, the rest is for a buffer’s
contents.  Rows in the display that are not occupied by a line from the buffer
are marked with tildes, “\type{~}”, and have a bluish color.  From the
display we can deduce that we haven’t loaded a file into \Vim, as the blue
status line says so.  Don’t worry about the anomalous {\tt line: 0 of 1} in
the status line, it will pick up as we begin inserting text into our buffer.
Finally, the black rectangle displays where point is.  Point is always just
before the symbol over which the black rectangle resides.

Let us begin by inserting some text into our buffer.  We’ll be writing an email
to our good friend, World.  To allow us to insert text, we must tell \Vim\ to
activate \Term{insert mode}, which is done by typing \KbdType{i}.  The display
changes, and \type{-- INSERT --} is displayed in the status line (see
\infigure[figure:vim insert-0]) to show that we’ve entered insert mode.
Once in insert mode, anything we type will go into the buffer.  Now type
\KbdType{H e l l o}\KbdKey{Space}\KbdType{W o r l d !}.  You should have
display similar to that in \infigure[figure:vim insert-1].

\placefigure
  [here]
  [figure:vim insert-0]
  {\Vim\ after entering insert mode.}
  {\externalfigure[vim insert-0]}

\placefigure
  [here]
  [figure:vim insert-1]
  {Typing \type{Hello World!} in \Vim.}
  {\externalfigure[vim insert-1]}

Now, let’s be courteous and ask World how she’s doing: type
\KbdType{H o w}\KbdKey{Space}\KbdType{a r e}\KbdKey{Space}\KbdType{y o u}%
\KbdKey{Space}\KbdType{h o l d i n g}\KbdKey{Space}\KbdType{u p ?}; the result
is in \infigure[figure:vim insert-2].

\placefigure
  [here]
  [figure:vim insert-2]
  {Inserting \type{How are you holding up?} in \Vim}
  {\externalfigure[vim insert-2]}

But wait a minute, we’d like the two sentences to appear on separate lines.  We
must go back to just after the \type{!} and insert a line||feed.  To do this,
we must first exit insert mode and return to \Term{command mode}.  This is done
by pressing the \KbdKey{Esc} key on your keyboard.  Once we’ve done that, we
must move point to said location.  We move point left by one symbol by issuing
the command \KbdType{h}.  Press the \KbdType{h} key until the black rectangle
is over the space following the \type{!}, as in \infigure[figure:vim move-1].
Once here, we enter insert mode again (\KbdType{i}) and press the \KbdKey{Enter} or
\KbdKey{Return} key on our keyboards.  The display is updated and should look like
that in \infigure[figure:vim insert-3].  Note that \Vim\ has removed the space
for us, it knows that we usually don’t want a space at the beginning of a new
line.

\placefigure
  [here]
  [figure:vim move-1]
  {Moving point left in \Vim.}
  {\externalfigure[vim move-1]}

\placefigure
  [here]
  [figure:vim insert-3]
  {Inserting a line||feed in \Vim.}
  {\externalfigure[vim insert-3]}

Furthermore, we want to keep our message short, so we’d like to trim the second
sentence somewhat.  Exit insert mode by pressing \KbdKey{Esc} again and move
over to the space just after the word \type{you} by issuing a series of
\type{l} commands (by pressing the \KbdType{l} key), which moves point one
symbol to the right each time.  \infigure[figure:vim move-2] displays the
result.

\placefigure
  [here]
  [figure:vim move-2]
  {\Vim\ after moving point right.}
  {\externalfigure[vim move-2]}

Once there, press the \KbdType{x} key to issue the command to delete the symbol
at point.  Do this until point is just before the \type{?} symbol, rendering
\infigure[figure:vim delete-1].

\placefigure
  [here]
  [figure:vim delete-1]
  {\Vim\ After Deleting Symbols}
  {\externalfigure[vim delete-1]}

Now we’ve finished typing our email.  We can save it to a file, continue
editing it, or a bunch of other tasks, but we’ll not delve any further into
the possibilities that our text editor presents us with.  We’ll, however,
summarize some of the lessons we’ve learned from this little text editing
task:

\startitemize[1]
  \item \Vim\ has editing modes, two of which are \Term{insert mode} and
    \Term{command mode}.
  \item \Vim\ has commands to move point relative to its current position,
    e.g., \type{h} and \type{l}.
  \item The \type{x} command|/|key allows us to remove an unwanted symbol at
    point.
  \item Although the contents of the buffer is only a long sequence of symbols,
    \Vim\ displays it as an array of lines.  Special symbols within the
    contents of the buffer tells \Vim\ where lines should be broken.  These
    symbols may be inserted by the user by pressing the \KbdKey{Enter} or
    \KbdKey{Return} keys on their keyboards and although conventions on exactly
    what symbols break lines differ across computer platforms, \Vim|<|and most
    other text editors|>|will understand them all, so the user only needs to
    know to type one of the designated keys on the keyboard.
\stopitemize

Here ends this intermission.  We’ll now continue describing more advanced text
editor functionality, most of which build upon the foundation of that already
described.  We’ll cover searching for substrings, editing modes, undo|/|redo,
and conclude with a discussion on command sets.


\section
  [search command]
  {Searching}

Scanning a text using only ones eyes and by moving point around one symbol at a
time is highly inefficient for larger texts.  Therefore, most text editors
provide a way to search for a substring in a buffer.  This is thus another
pattern matching problem and, as we’ve already seen in \inchapter[patterns],
regular expressions work out great for this task.  Many text editors in fact
utilize regular expressions for searching.

The search operation can be broken down into a sequence of steps:

\startenumerate
  \item The user invokes the \Term{search} command.

  \item The text editor provides a mean for user input for reading the desired
    pattern to search for.

  \item Once the user has terminated the pattern, the text editor feeds the
    pattern to a pattern matcher generator|<|making sure to mark the pattern
    for submatch addressing|>|and the contents of the buffer to the resulting
    pattern matcher.

  \item If the pattern matcher answers “yes”, the text editor extracts the
    resulting submatch addresses and moves point to the beginning of the match,
    i.e., the first component of the submatch address of the pattern.
    Otherwise, it leaves point where it was and notifies the user that it has
    failed in finding the given pattern.
\stopenumerate


\section{Editing Modes}

As discussed earlier, some editors have what is known as \DefineTerm{editing
modes} and behave differently depending on what mode they’re currently in.
Modes differ by how they interpret the keys that the user types on their
keyboard.  Some typical modes include

\startterm{Insert Mode}
  In this mode, pressing a key on the keyboard inserts the symbol associated
  with that key.  This operation is commonly referred to as \Term{self-insert}
  as the key pressed inserts “itself”, i.e., the symbol associated with that
  key.  Some key sequences including modifier keys such as \KbdKey{Ctrl} and
  \KbdKey{Meta}/\KbdKey{Alt} may have other effects.
\stopterm

\startterm{Command Mode}
  In this mode, keys are associated with a command.  We saw, in our intermission
  on editing with \Vim, that in command mode that typing \KbdType{h} would
  invoke a command that moved point one symbol to the left.  Having a command
  mode allows advanced editing commands to be associated with a single
  keystroke, which speeds up many editing tasks.
\stopterm

\startterm{Command||line Mode}
  The final mode we’ll discuss is a mode where more advanced commands are
  executed.  In command||line mode, a text editor will read longer commands
  from the user.  An example of such a command is the search command discussed
  in \insection[search command], which reads a pattern provided by the user.
\stopterm

Many editors combine the first two modes and associates key sequences
involving modifier keys to execute the commands provided in command
mode; \EMACS\ works this way.


% TODO: should we describe how it works internally (here or later?)
\section{Undoing and Redoing Changes}

Any self||respecting text editor soon realizes that the weakest link in the
editing chain of command is its user.  Therefore, they provide a way for the
user to undo alterations made to the text during the editing session.  To
provide additional protection against an unwanted undo, a way to redo
previously undone changes is often also available.


% TODO: this needs to be written...suck
%\section{Command Sets}

\stopcomponent
