\startcomponent thesis/chapters/regex-and-text-editors

\project masters-project
\product thesis

\chapter
  [regexes in text editors]
  {Using Regular Expressions in Text Editors}


The outset of this manuscript was to introduce the theoretical foundations for
regular expressions and text editors.  Another goal was to look at the
connection between the two, and specifically how regular expressions can be
used effectively in a text editor.  We’ll be looking at the command||sets of
two editors, \Vim\ \cite[Moolenaar02,Oualline01]\ and \SAM\ \cite[Pike87].
These are both solid editors that use regular expressions for more tasks than
most.  

This chapter will look especially at the parts of the command sets of either
editor that relate to regular expressions.  We’ll try to identify (1) what
commands are useful to the user but also any problems that these commands may
suffer in their current form and (2) commands that seem to be missing.  Once we
have gathered this information we’ll design a command set for our new editor
that will be the center of attention from now on in this manuscript, named
\NED.

As already stated, two editors will be investigated, namely \Vim\ and \SAM.
The reasons for choosing \Vim\ are many:  it has a long history and uses a lot
of the same ideas that were present in its predecessors, such as \Qed, \ED, and
\VI;  its use of regular expressions is almost ubiquitous throughout the
editor.  \Vim\ is also a very popular editor, e.g., see \cite[LJ127], making it
a good choice for investigation.

When the developers of \PlanNine\ needed a new editor for their operating
system, \SAM\ was the result \cite[Pike95].  It was an attempt at creating a
better version of \ED, \VI, and \JIM, much as \PlanNine\ itself was an attempt
at a better \UNIX.  It, as its predecessors, had a command||set that was
designed around the application of regular expressions for performing textual
transformations.  It did, however, introduce a lot of new ideas, trying to free
itself from some of the restrictions that had been imposed by earlier editors.

The discussion assumes some knowledge the syntax of \Vim’s regular expressions.
\inappendix[vim regular expression syntax]\ contains a complete reference to
\Vim's regular expression syntax, for those unfamiliar with it.  Examples have,
however, been designed so as to introduce as few new concepts as possible, so
by all means do read on.  If you have access to \Vim\ or \SAM\ you may want to
run these editors alongside reading this manuscript to see the effects of the
commands that we’re about to introduce “live” so to speak.  Doing helps
learning, as they say.

% TODO: describe this chapter's composition
% This chapter begins by introducing the command set of \Vim.


\section{Search Commands}

\Vim\ allows users to express search patterns using regular expressions.  The
search command is named \type{/} and takes a regular expression as its
only parameter:

\starteditorinput
\type{/}\<pattern>
\stopeditorinput

An example of the search command is displayed in \infigure[vim:search-1] where
the user has entered a search for the word “sentence” on the command input line
at the bottom of the display.  The command typed was \KbdType{/ \ < s e n t e n
c e \ >}.  (Please see \inchapter[vim regular expression syntax] for the
meaning of the surrounding \TypedRegex{\<...\>}.)

The matching word is highlighted while the user is still entering the regular
expression, making it easy to track what text matches the pattern.  Once
the search is ended, by pressing \KbdKey{Enter}, point is placed at the
beginning of the matched text~--~\infigure[vim:search-2].  (The black rectangle
is now display using reverse||video, to make it easier to read the text beneath
it.)  In addition, all matches have now been highlighted to make it easier to
see where other matches to the search||pattern are.

\placefigure
  []
  [vim:search-1]
  {Entering a search pattern in \Vim.}
  {\externalfigure[vim:search-1]}

\placefigure
  []
  [vim:search-2]
  {Displaying matches to a pattern in \Vim.}
  {\externalfigure[vim:search-2]}

Invoking the search command is done in the same fashion in \SAM\ as in \Vim.

\section{Movement Commands}

Related to search commands are movement commands, as a search moves point to a
new position in the buffer where a match was found.  As \Vim\ treats search
commands as movement commands we may use them to do complex edits.  We may, for
example, use the delete command \type{d} together with a search command to
delete from point to the end of the current sentence.  The \type{d} command is
defined as follows:

\starteditorinput
\type{d}\<motion command>
\stopeditorinput

A \<motion command> is a command that moves point in some way.  The \type{/}
command fits this description.

Let us issue a delete command, using a search as the \<motion command>
argument.  \infigure[vim:movement-1] is the state the buffer is in before the
command is executed.  The complete command that is then executed is \KbdType{d
/ [ . !  ? ] \ s \ +}\KbdKey{Enter}.

\placefigure
  []
  [vim:movement-1]
  {Using search together with delete (before).}
  {\externalfigure[vim:movement-1]}

After the delete||using||search command is executed, the buffer will be as in
\infigure[vim:movement-2].  All text up to the matched text has been deleted.
Thus our sentence that was too long\footnote{Or perhaps it was just fine in
length, thus making the comment at the end of it redundant} is now just right.

\placefigure
  []
  [vim:movement-2]
  {Using search together with delete (after).}
  {\externalfigure[vim:movement-2]}

{\SAM}s command||set doesn’t include any commands that take a \<motion> as an
argument, thus this discussion doesn't apply to it.


\section{Addressing Commands}

Commands in \Vim\ may be directed to only address a certain set of lines in the
buffer.  The addressing is done by giving a continuous range of lines, where
each end||point is defined by a \DefineTerm{line||number specification}.  Such
a specification may be given in many different ways, but it’s enough for our
discussion to known that it’s given by a pair of {\em addresses}, separated by
commas (,):

\startbnfgrammar[]
  <line-number specification>: <address> | <address> "," <address>
\stopbnfgrammar

We actually allows ranges to span only one line and in this case specifying a
sole address suffices.

An \DefineTerm{address} is either (a) a number, representing the line with that
line||number; (b) a relative number, representing a line with a line||number
relative to the current number; or (c) a pattern, representing the line||number
of the next line that contains a match of that pattern:

\startbnfgrammar[]
  <address>: <number> | <relative number> | "/" <pattern> "/"
  <relative number>: "+" <number> | "-" <number>
\stopbnfgrammar

If we would like to delete the next empty line and the two lines following it,
we may give the delete command a line||number specification that will match
this specification.  The exact expression for this range is

\starteditorinput
\type{/^$/,+2}
\stopeditorinput

A command is given a range by invoking them from the \DefineTerm{command line}.
This is done by executing the colon (\type{:}) command.  Following the colon is
the line||number specification and then the name of the command to execute:

\starteditorinput
\type{:}\<line||number specification>\<command>
\stopeditorinput

The buffer before the edit we are about to perform is shown in
\infigure[vim:addressing-1].

\placefigure
  []
  [vim:addressing-1]
  {Addressing the delete command (before).}
  {\externalfigure[vim:addressing-1]}

The command to be executed is thus \KbdType{: / ^ $ / , + 2 d}\KbdKey{Enter}.
The addressed command is invoked by the colon (\type{:}) command.  The colon is
followed by the line||number specification that was introduced above.  Finally,
the \type{d} is still the name of the delete command, which completes our
command.  \infigure[figure:vim:addressing-2]\ displays the state of the buffer
after the command has been executed.

\placefigure
  []
  [figure:vim:addressing-2]
  {Addressing the delete command (after).}
  {\externalfigure[vim:addressing-2]}

Addressing commands works in much the same way in \SAM\ as in \Vim.


\section
  [repeating commands]
  {Repeating Commands}

% TODO: better to use pattern or regular expression in this chapter?
\Vim\ provides means to repeat a given command an a set of lines containing
matches of a pattern.  The meta||command that provides this functionality is
known as the \type{g} command, for \quotation{global}.  Its definition is

\starteditorinput
\type{:g/}\<pattern>\type{/}\<command>
\stopeditorinput

The \type{g} command has many uses and we can in no way show you all.  We do
however provide you with a global command that deletes all empty lines in the
buffer.  We execute the global command \KbdType{: g / ^ $ / d}\KbdKey{Enter} on
the buffer displayed in \infigure[figure:vim:repeating-1].  Again, the pattern
given as an argument to the global command is delimited by slashes.  The
command to execute on the lines chosen is the delete command \type{d}, which
follows the pattern specification.

\placefigure
  []
  [figure:vim:repeating-1]
  {Repeating a command in \Vim\ using the \type{g} command (before).}
  {\externalfigure[vim:repeating-1]}

The resulting buffer is shown in \infigure[figure:vim:repeating-2].  

\placefigure
  []
  [figure:vim:repeating-2]
  {Repeating a command in \Vim\ using the \type{g} command (after).}
  {\externalfigure[vim:repeating-2]}

The repetition of commands is an area where \SAM\ finally begins to differ from
\Vim\ and ancestors they have in common.  \SAM\ doesn’t have the \type{g}
command.  Instead, \SAM\ has a much more flexible command called \type{x} for
{\em extract}.  While the \type{g} command in \Vim\ only allows commands to
operate on a per||line basis, the \type{x} command in \SAM\ allows commands to
operate on a per||match basis.

To anyone familiar with \UNIX\ tools in general, where almost everything is
done on a per||line basis, this may seem odd and perhaps even controversial.
Once you begin to understand the potential of per||match operation, however, it
turns out that this opens up for a lot of new possibilities.

To continue the discussion on repetition commands in \SAM, we must first
introduce a new notion.  \SAM\ uses a much more flexible definition for
point\footnote{point is actually known as \DefineTerm{dot} in the \SAM\
documentation, but we’ll stick to using the same name we been using so far.}.
The point in \SAM\ may not just be a position in the contents of the buffer,
but a range.  We’ll, however, only redefine one of the operations that have
previously been introduced, so as to match this ranged point\footnote{It’s of
course a misnomer for a point to be a line, but we aren't discussing text
editing from a geometric point of view.  See instead \cite[Miller99,Miller02]\
for such a discussion.}.

% TODO: how are ranges indexed?
\startdefinition
  The \DefineTerm{move||to operation} of the \SAM\ text editor moves point to a
  new position in the buffer:

  \startnathequation
    \Fmoveto\colon \Buffers×\naturalnumbers → \Buffers, \\
    \Fmoveto((w, p), ⟨q_0, q_1⟩) = (w, ⟨q_0', q_1'⟩),
      \qquad \wall q_0' = \min(\min(q_0, q_1), |w| + 1), \\
                   q_1' = \min(\max(q_0, q_1), |w| + 1). \return
  \stopnathequation
\stopdefinition

Equipped with this advanced definition of point and \Fmoveto, we are able to
explain how the \type{x} command operates.  The \type{x} command takes two
arguments, a pattern and a command, and for each match of the pattern, it sets
point to the matched portion of the buffer and executes command.  This command
may in turn alter the contents of the buffer within point, or it may be another
\type{x} command to do further extractions within the matched portion.

The \type{x} command is thus invoked in the following manner:

\starteditorinput
\type{x/}\<pattern>\type{/}\<command>
\stopeditorinput

An example usage of the \type{x} command would be to change every instance of
the word “acceptable” to “unacceptable”.  To do so we must first introduce
another command available in \SAM, the insert command \type{i}:

\starteditorinput
\type{i/}\<text>\type{/}
\stopeditorinput

The \type{i} command inserts \<text> before point without changing it.  Thus,
to swap charges from the positive to the negative, we execute the following
command:

\starteditorinput
\type{x/acceptable/i/un/}
\stopeditorinput

\SAM\ has three more commands that relate to the \type{x} command: the
\type{y}, \type{g}, and \type{v} commands.

The \type{y} command in \SAM\ is like the \type{x} command, with the defining
difference being that it matches between adjacent matches of its argument
pattern:

\starteditorinput
\type{y/}\<pattern>\type{/}\<command>
\stopeditorinput

This can be useful in fixing our above example of the \type{x} command, where
the word “unacceptable” will actually be changed to “ununacceptable”, which is
quite unacceptable.  The solution is to use \type{y} to skip over the instances
of the word \quotation{unacceptable}:

\starteditorinput
\type{y/unacceptable/x/acceptable/i/un/}
\stopeditorinput

Pretty darn neat.

We mentioned earlier that there were commands named \type{g} and \type{v} as
well.  These two commands are each others complement and they acts as guards
for the command that has been given as an argument:

\starteditorinput
\type{g/}\<pattern>\type{/}\<command>
\stopeditorinput

The semantics of the \type{g} command is such that if the pattern matches
within point, the given command will be executed.  The semantics of the
\type{v} command is to do just the opposite, execute the command if the pattern
doesn’t match within point.

As an example, consider the file in \infigure[figure:pike bibliography].  Say
that we wish to extract any bibliography entry that refers to \SAM\ itself.  A
bibliography entry can be matched by the regular expression
\TypedRegex/@[a-z]+{.*\n([^}].*\n)+}\n/.  That was perhaps a mouthful, so well
go through it step||by||step.  The \TypedRegex/@[a-z]+{/ matches the beginning
of the bibliography entry.  It’s then followed by some information that we
don’t care much about, matched by \TypedRegex{.*\n} to the end of the line
(\TypedRegex{\n} matches the end of a line).  This is followed by one or more
non||empty lines that don’t begin with a curly brace (\type|}|), thus
\TypedRegex/([^}].*\n)+/ fits the bill.  The bibliography entry ends with a
line that only contains a single curly brace: \TypedRegex/}\n/.

\placedescribedfigure
  []
  [figure:pike bibliography]
  {A set of bibliography entries in \BIBTEX\ format \cite[Pataschnik88]\ taken
   from a larger set due to \Name{Rob}{Pike}.}
  {\typefile{../thesis/data/pike-bibliography.bib}}

We can feed this to the \type{x} command and add a guard that check that the
matched bibliography entry mentions \SAM\ in the title, before we display it
using the \type{p} command:

\starteditorinput
\type:x/@[a-z]+{.*\n([^}].*\n)+}\n/g/  title = .*sam/p:
\stopeditorinput

We haven’t mentioned the \type{p} command yet, but there isn’t much to say
about it.  It simply displays the contents of point, in this case being

\starttyping
@article{Pike87a,
   author = {Rob Pike},
   title = {The Text Editor sam},
   journal = {Software - Practice and Experience},
   volume = {17},
   number = {11},
   pages = {813--845},
   month = nov,
   year = {1987}
}
\stoptyping

This concludes our discussion on repeating commands.  This topic is probably
the hardest one to grasp of the ones introduced in this chapter, and to get a
feel of how to use them effectively it’s probably best to sit down and run some
interactive sessions with the editors.  Don’t let their potential power scare
you off; instead harness it to achieve advanced textual transformations.


\section{Substitution Command}

One of the most useful commands in \Vim\ is the substitution command \type{s}.
This command takes a pattern and a substitution specification as arguments and
then searches for matches of the pattern.  Whenever a match is found, it’s
substituted with what was given as the second argument.  This substitution
specification can be rather advanced|<|one might even consider it a
sub||language in itself, as it allows so many things to be done|>|but usually
it’s simply another string to replace the match with.

The definition of the \type{s} command is

\starteditorinput
\type{:s/}\<pattern>\type{/}\<substitution specification>\type{/}
\stopeditorinput

and the exact semantics of it is that it will search for matches to the given
pattern within the current line, i.e., the line containing point, and replaces
the first occurence with the given \<substitution specification>.  As
previously stated, we won't give a complete grammar for this specification,
only a partial one:

\startbnfgrammar[]
  <substitution specification>: <string>
\stopbnfgrammar

For a complete grammar and explanation, see
\inbnfgrammar[vim regular expression syntax:grammar:substitution specification]
and
\intable[vim regular expression syntax:table:substitutions].

As an example we would like to substitute misspellings of “acceptable” in the
buffer displayed in \infigure[figure:vim:substitution-1].

\placefigure
  []
  [figure:vim:substitution-1]
  {Substituting a string in \Vim\ (before).}
  {\externalfigure[vim:substitution-1]}

The command executed is
\KbdType{: s / a c c e p t i b l e / a c c e p t a b l e}\KbdKey{Enter}.  The
pattern is separated from the replacement by a slash (\type{/}).  The resulting
buffer can be seen in \infigure[figure:vim:substitution-2].

\placefigure
  []
  [figure:vim:substitution-2]
  {Substituting a string in \Vim\ (after).}
  {\externalfigure[vim:substitution-2]}

Remember: this is a very simple instance of the use of the \type{s} command.  A
lot more advanced substitutions can be made with just a little more effort.

\SAM\ has the \type{s} command with a similar syntax, but with different
semantics.  In \SAM, \type{s} looks for a match to the pattern within the
contents of point and substitutes it with the replacement string.  The \type{s}
command isn’t as useful in \SAM\ as in \Vim, as there are other (often better)
ways to achieve the same effect.  It’s included mostly for convenience, as
predecessors to \SAM\ had this command, and it would seem strange to some not
to have it available.

% TODO: conclude this section

\section
  [problems]
  {Problems, Solutions, and Conclusions}

So, now that we have seen some of the commands in \Vim\ and \SAM\ that use
regular expressions, it’s time to discuss some of the issues associated with
these commands and their possible solutions.

As the previous sections have shown, \Vim’s command set is very line||centric,
where most commands work on a set of lines as the smallest unit.  This works
well for many tasks, but it isn't unusual that one would like for commands to
work on an arbitrary structure of the buffer.  Examples of this can be seen in
the section on repeating commands (\insection[repeating commands]) where we
defined our structure to be a \BIBTEX\ bibliography entry.  While this was a
structure larger than a single line, we can think of instances where we would
like to operate on smaller units than a line, e.g., to delete each occurrence
of a word.  \SAM\ provides a very interesting solution to this problem.  It
seems that the set of four commands, \type{x}, \type{y}, \type{g}, and
\type{v}, associated with the repetition of commands, are enough to make the
most daunting editing tasks simple to describe to the text editor.
% TODO: have we shown this?

The problem with \SAM\ is that it doesn’t go far enough.  While it solves
problems with the command set, the use of regular expressions could still be
made easier.  The main problem with regular expressions in either text editor
is that their syntax is too cryptic, making it hard to express advanced
instances of them.  This is the problem that we’ll solve in the next chapter,
where we introduce a new regular expression syntax that will make it easier for
the user to create a set of often used regular expressions for easy reuse.

% TODO: further discuss other problems

\stopcomponent
