\startcomponent thesis/chapters/the-real-world

\project masters-project
\product thesis

\chapter
  [the real world]
  {Regular Expressions and The Real World}


This chapter is about using regular expressions to solve problems that occur in
{\em The Real World}.  We’ll take a look at how we transfer regular expression
syntax to a limited alphabet, common extensions to this syntax, and also at
some applications of regular expressions.


\section{Operators and Escaping}

In \insection[definitions:regular expressions] we gave a syntax for regular
expressions using the symbols $|$, $∗$, $($, and $)$|<|which were not part of
any alphabet used|>|as our operators.  While we can reason about regular
expressions using these symbols without ambiguity, in {\em The Real World} our
alphabet is generally limited in some sense, e.g., to the \ASCII\ or \Unicode\
character set (\insection[definitions:character encodings]).

If we represent the operator symbols by their most logical mapping onto the
\Unicode\ character set, namely \CodePoint{007C} (\type{|}), \CodePoint{002A}
(\type{*}), \CodePoint{0028} (\type{(}), and \CodePoint{0029} (\type{)}), we
have no way of representing these characters as themselves in a regular
expression.  Thus, we need some way to put these symbols in a regular
expression without them being interpreted as operators.  As we cannot generally
assign new code points to the same characters as we are going to use as our set
operators (the \Unicode\ Standard simply will not allow this), we have to
define some kind of escape method that disables the interpretation of a
character as an operator.

An almost universal non||standard in computing is to perform character escaping
by preceding the character in question with a backslash (\type{\}) and to use a
sequence of two backslashes (\type{\\}) to represent a single one.  This is the
solution employed by most computer software that deals with regular expressions
as well.  There are alternatives schemes used, but they are mostly found in
non||interactive software, such as in defining lexical scanners
\cite[Paxson95]\ or programming language grammars \cite[Forsberg03].

To complicate matters, the interpretation of the escaping operation is not
always as suggested above, i.e., to {\em not} treat the character as an
operator, but instead to treat the next character as an operator instead of
as a normal symbol.  Therefore, when we want to pass the regular expression
$(a \Alt b \Alt c)$ to the \UNIX\ command \Command{grep}, we write it as
\TypedRegex{\(a\|b\|c\)}.  This is known as the
\index{Regular Expression Syntax+Basic}\Term{Basic Regular Expression
Syntax}, or \BRE.  However, when we want to pass the same regular
expression to the \UNIX\ command \Command{egrep} \cite[McIlroy86,POSIX92], we
instead write it as \TypedRegex{(a|b|c)}.  This is known as the \index{Regular
Expression Syntax+Extended}\Term{Extended Regular Expression Syntax}, or \ERE.
If we instead wanted to pass the regular expression matching the actual
characters comprising it, we would swap them.

\placeintermezzo
  []
  [intermezzo:grep and egrep]
  {Explanation of the \Command{grep} and \Command{egrep} commands.}
  \startframedtext
    The \UNIX\ commands \Command{grep} and \Command{egrep} deserve some further
    discussion.  They are utilities for finding lines of text matching a given
    regular expression in a set of files.  They are invoked at a computer
    terminal running \UNIX\ as follows:
    \startlines
    \type{% grep} \<regular expression> \<file name>\dots
    \stoplines
    Here the contents of each file \<file name> (the ellipsis signifies that
    any number of names of files may follow) will be matched against the
    regular expression \<regular expression>.  That is to say, the contents is
    not treated as a single unit, rather multiple starting points are tried, so
    that the finite automaton that \Command{grep}/\Command{egrep} employs will
    be called upon at all possible places where it has yet to be tried for a
    match.  A simpler way of stating this is to say that \<regular expression>
    is augmented by prefixing it with the regular expression \TypedRegex{.*} so
    that it becomes \TypedRegex{.*}\<regular expression>.  The finite automaton
    is also told to report an accepting configuration as soon as an accepting
    state is reached, i.e., not all input must be consumed for the automaton to
    be successful. The \TypedRegex{.} operator matches anything in the input,
    and is discussed further in \insection[any operator].  Searching text for
    patterns are covered in \inchapter[patterns], \at{page}[patterns].
  \stopframedtext

Both \BRE\ and \ERE\ are defined in the \POSIX\ standard \cite[POSIX92]\ and
are used by many tools belonging to computer systems adhering to this standard.
We will from now on use the \ERE\ syntax whenever we discuss regular
expressions used on a computer, unless otherwise stated.


\section{Common Extensions}

The three operators|<|with concatenation being implicit|>|and subexpression
grouping that we have defined for regular expressions are powerful enough to
represent any finite automaton.  It is, however, painful to write any but the
simplest of regular expressions using only these operators.  For this reason, a
lot of extensions to the regular expression grammar have been made over time to
ease their use in everyday application.  This section tries to highlight some
of the most often occurring ones.  It is, again, vital to note that none of
these extensions increase the expressive power of the underlying regular
expression (they all maintain the regularity of it), only making for more
succinct ones.  But, as \Name{Paul}{Graham} will tell you \cite[Graham02],
succinctness {\em is} power.

\subsection
  [any operator]
  {The “any character” character that matches any character.}

The character \TypedRegex{.} stands for the notion of {\em any character}.
Formally, for an alphabet $Σ$, \TypedRegex{.} matches the language
$L = Σ^1$, i.e., the set of all strings of length one over the alphabet
$Σ$. This means that the regular expression \TypedRegex{...} will match any
string of length three in the input alphabet.  It is an idiom to match a
sequence of any number of characters using \TypedRegex{.*}.

\subsection
  [character ranges]
  {Character ranges provide a way to express complex ranges of characters.}

It’s often useful to have a set of characters that may be accepted at a given
point in a regular expression.  Therefore, the sequence
\type{[}$a_1 a_2 \dots a_n$\type{]}, for $n ≥ 1$ and where each $a_1$,~$a_2$,
\dots,~$a_n$ is one of

\startitemize
  \item a single character, $x$, matching itself;

  \item the sequence $x$\type{-}$y$, where $x$ and $y$ are characters and
    $x ≤ y$, matching any character $c$ in the range
    $\Fcodepoint(x) ≤ c ≤ \Fcodepoint(y)$; or

  \item a \POSIX\ character class,
\stopitemize

is equivalent to the regular expression $(a_1 \Alt a_2 \Alt \dots \Alt a_n)$,
where each $x$\type{-}$y$ range has, in turn, been expanded to
$(x \Alt \dots \Alt y)$ and each \POSIX\ character class has been expanded to
the union of all characters in the given class.

Finally, as a special case, if $a_1$ is \type{^}, the character range will
represent the negation of its contents.  Thus, if the alphabet is $Σ$, the
character range will match

  \[Σ \setminus \{a_1, a_2, \dots, a_n\},\]

where each $x$\type{-}$y$ range and \POSIX\ character class among $a_1$,~$a_2$,
\dots,~$a_n$ has been expanded in||place to the characters they contain.

We haven’t explained what \POSIX\ character classes are yet, so let’s do so
now.  A \index[POSIX character class]{\POSIX\ character class}\Term{\POSIX\
character class} is an expression delimited by \type{[:}, \type{:]} pairs
inside a character range.  The possible values are

\startitemize[columns,four]
  \nop \type{[:alnum:]},
  \nop \type{[:alpha:]},
  \nop \type{[:blank:]},
  \nop \type{[:cntrl:]},
  \nop \type{[:digit:]},
  \nop \type{[:graph:]},
  \nop \type{[:lower:]},
  \nop \type{[:print:]},
  \nop \type{[:punct:]},
  \nop \type{[:space:]},
  \nop \type{[:upper:]},
  \nop \type{[:xdigit:]}.
\stopitemize

These expressions match a locale||dependent set of characters/code points.
This fact makes them very useful when the exact set of characters is
well||defined in this way.

\placeintermezzo
  {Further extensions to character ranges defined by \POSIX.}
  \startframedtext
    \POSIX\ also adds what is known as \DefineTerm{collation elements} and
    \DefineTerm{equivalence classes} to character ranges, which are useful for
    internationalization purposes.  Their syntax and semantics are, however,
    ill||defined and are rarely, if ever, used.  Therefore, we will simply
    ignore them.
  \stopframedtext

There are a few border cases in this mini||language of regular expressions to
be dealt with.  For example, how do we represent \type{-} and \type{]} in it?
We will, however, not go into details on how different implementations solve
this, as few agree as to how it should be done and these specifics are
irrelevant to our discussion.

It should be noted that these kinds of character ranges were thought of when
most of the computer world were using the \ASCII\ character set, not the
\Unicode\ one.  \Unicode\ complicates matters somewhat, and it is often hard to
use $x$\type{-}$y$ ranges successfully.  A common idiom, under \ASCII, is to
use the expression \TypedRegex{[A-Z]} to match an upper||case letter.
Under \Unicode, the notion of upper||case letter is not as simple as “all the
characters between ‘A’ and ‘Z’\thinspace”.  There are title||cased letters,
many different scripts not all based on the Latin one, and some scripts do not
even have the notion of upper||case and lower||case.  Thus, some \cite[Wall02]\
consider character ranges a thing of the past, best restricted to cases where
the exact range of characters to match is well defined\footnote{Such as in an
\RFC\ 822 \cite[RFC822]\ compliant header field, where only \ASCII\ characters
between codepoints 33 through 126, excluding colon at 58, are allowed by the
standard.}.

To complicate|<|or perhaps simplify|>|matters somewhat,
\TypedRegex{[}\dots\TypedRegex{]} works the same way in both \BRE\ and \ERE,
i.e., no escaping is necessary in either syntax.

\subsection
  [iteration shorthands]
  {The closure operator, \TypedRegex{*}, is arguably the most powerful operator
   of regular expressions, but additional control over it makes it easier to
   use.}

Often you will want the closure of a regular expression, and thus
\TypedRegex{*} fits the bill very well.  It does, however, happen that you want
a bit more fine||grained control over exactly how many times the concatenation
may take place.  A syntax has grown out of this need to represent some of the
most often occurring cases and they are listed here.

\startitemize
  \item $r$\TypedRegex{?} is equivalent to $ε \Alt r$, i.e., “zero or one
  repetitions of $r$”.

  \item $r$\TypedRegex{+} is equivalent to $rr^∗$, i.e., “one or more
  repetitions of $r$”.

  \item $r$\TypedRegex/{/$n$\TypedRegex{,}$m$\TypedRegex/}/ is equivalent to

    \[\underbrace{\mathstrut rr\dots r}_{n \text{ times}}
      \underbrace{\mathstrut (ε|r)(ε|r)\dots(ε|r)}_{m - n\text{ times}},\]

    i.e., “$n$ to $m$ repetitions of $r$”.  Either $n$ or $m$ (and sometimes
    both) can often be left out.  If $n$ is left out, it defaults to $0$.  If
    $m$ is left out, it defaults $∞$, and the right sequence in the formula
    above is replaced by $r^∗$.  Thus, if both $n$ and $m$ are left out,
    $r$\TypedRegex/{,}/ (where the comma (\TypedRegex{,}) may often be left
    out) is equivalent to $r$\TypedRegex{*}: the closure of $r$.  Finally, if
    $m$ and the comma (\TypedRegex{,}) are left out, leaving only $n$, then $m$
    is taken to be equal to $n$.
\stopitemize

The \TypedRegex/{/$n$\TypedRegex{,}$m$\TypedRegex/}/ operator is generally not
available in tools that use the \BRE\ syntax.  When it is, what’s escaped
often differs.  Some escape only the initial \TypedRegex/{/, others escape both
delimiters.  A further discussion of the theoretical ramifications of the
introduction of this operator can be found in \cite[Kilpelainen03].

\subsection
  {Technically only useful when doing pattern matching, assertions provide a
   way to match the position of the input cursor.}

\Note The following operators will make more sense when we look at how regular
expressions are generally applied to the task of searching text for matches.
We include them here, however, as they fit better under the current section
heading.  You may skim over the contents of this subsection and return to it
when necessary.

An assertion operator makes some kind of assertion about the current position
in the input that is being matched.  When searching an input string, regular
expressions are often tried for a match at every possible point in the input
and thus the regular expression \TypedRegex{second} will match the substring
starting at the seventh position and ending at the twelfth position in the
string \TypedString{first \REMatch{second} third}.

This is where assertions are useful.  They match a position in the input,
rather than a character. \intable[table:assertion operators] lists the
operators along with the position that they match.  There are some alternatives
to this syntax and further extensions to this scheme; their purpose often being
to disambiguate the meanings of the ambiguously defined operators.

\placetable
  []
  [table:assertion operators]
  {Assertion Operators}
  \starttable[|l|lp(27em)|]
  \HL
  \NC \bf Operator \NC \bf Matches \NC\AR
  \HL
  \NC \TypedRegex{^} \NC
    Start of the input or at start of any line within it: \TypedRegex{^here}
    matches \TypedString{\REMatch{here} or there} but not
    \TypedString{adhere to}. \NC\AR
  \NC \TypedRegex{$} \NC %$
    End of the input or at end of any line within it: \TypedRegex{there$} %$
    matches \TypedString{here or \REMatch{there}} but not
    \TypedString{there’s}. \NC\AR
  \NC \TypedRegex{\b} \NC
    At a word boundary.  A word boundary is usually defined as the position
    right before an alphanumeric character following (a) white||space, (b) the
    beginning of a line, (c) the beginning of the input, or right after an
    alphanumeric character followed by (d) the end of a line, or (e) the end of
    the input.  So \TypedRegex{\bword\b} matches \TypedString{this is a
    \REMatch{word} boundary} but not \TypedString{expressive words are
    preferable}. \NC\AR
  \HL
  \stoptable

\subsection
  {The one exception that breaks everything: back referencing.}

We actually lied a bit when we said that all of the commonly found extensions
maintained the regularity of the underlying regular expression.  There is one
extension that greatly complicates matters~--~the back reference.

\index{back reference}\Term{Back references} are a way to refer to what has
already been matched by a subexpression of a regular expression.  Generally,
any subexpression demarcated by \TypedRegex{(}\thinspace\dots\TypedRegex{)} is
up for consideration.  We introduce the operator \TypedRegex{\}$n$, which will
match whatever the $n$th subexpression, counting left parentheses
(\TypedRegex{(}) from the left, matched.  An example of this is the regular
expression \TypedRegex{([a-z]+) \1}, which|<|in a limited sense|>|matches
repeated words in the input, such as any of the \TypedString{the the} pairs in
\TypedString{... any of the the the ...}.

This may seem like a great extension that has a lot of uses, such as the one
exemplified in the previous paragraph.  Well, it turns out that adding this
extensions does {\em not} maintain the regularity of the affected regular
expression, and makes the problem of matching it against an input \NP||complete
\cite[Aho90].  Without going into details on \NP||hardness and
\NP||completeness, this means that the problem of matching a finite automaton
created from a regular expression against some input string goes from a problem
solvable in time proportional to the size of the automaton and the length of
the string, to a problem which for many kinds of automata and inputs probably
doesn’t have a solution that terminates in the course of the life expectancy of
our galaxy (or our universe for that matter).

Beyond the problem of back referencing ruining a lot of nice properties of
regular expressions, it’s generally not as useful as initially thought.  The
repeated word example given above is one of a very small set of useful cases
where back referencing is useful.  However, as it is
written|<|\TypedRegex{([a-z]+) \1}|>|it will miss a lot of cases that can
appear.  What if the repeated words are not separated simply by a space?  This
will have to be dealt with in some much more forgiving manner.  Also, a lot of
tools, such as \Command{grep} won’t match across line||boundaries, which
means that the “\type{the}\crlf\type{the}” in this manuscript won’t be matched
correctly.

Overall, the back||referencing extension is not as useful as it may first seem
and should probably never have made it into the \POSIX\ standard.  Study of the
properties of regular expressions with back referencing is lacking, most likely
due to their \NP-completeness \cite[Aho90].

Back referencing first appeared in \SNOBOL\ \cite[SNOBOL64], a programming
language designed for processing text and designing text editors.  It later
appeared in several \UNIX\ commands, such as \Command{ed}
\cite[McIlroy86,POSIX92]\ and \Command{grep}.


\section{A Concluding Example or Two}

Using the extensions listed above, we may in a concise manner express the
regular expression that denotes the automaton for matching floating point
numbers discussed in \inexample[definitions:example:epsilon decimal number].
The result can be seen in \inregex[regex:floating point number].

\placeregex
 []
 [regex:floating point number]
 {Floating point numbers}
 {\midaligned{\TypedRegex{[+-]?[0-9]*([0-9]\.|\.[0-9])[0-9]*([eE][+-]?[0-9]+)?}}}

A bit more general than our initial specification,
\inregex[regex:posix floating point number] matches floating point number using
\POSIX\ character classes for matching the digits constituting the number.
This is certainly a regular expression you wouldn’t want to type in often.
The \POSIX\ character classes make the intent of the character range clear, yet
obfuscates the regular expression as a whole and it’s questionable if its
advantages outweigh its disadvantages.  Just the fact that it was necessary to
break it up into two lines in this manuscript is telling\dots

\placeregex
  []
  [regex:posix floating point number]
  {Floating point numbers using \POSIX\ character classes}
  {\midaligned
     {\TypedRegex{[+-]?[[:digit:]]*([[:digit:]]\.|\.[[:digit:]])[[:digit:]*}}
   \crlf
   \midaligned
     {\TypedRegex{([eE][+-]?[[:digit:]]+)?}}}

\stopcomponent
