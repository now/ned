\startcomponent thesis/chapters/conclusion

\project masters-project
\product thesis

\chapter
  [future work]
  {Future Work}


\section{Future Work~--~Ideas That Deserve Further Study}


\subsection{The Pattern||matching Library}

Although the pattern||matching library works well and produces seemingly small
and fast automata, experiments confirming this would add merit to its
existance.  Data recorded in earlier tests of {\TNFA}s \cite[Laurikari01]\
suggest that {\TNFA}s compare well with other finite automata.

There remain ways to make things better, though.  The current implementation
uses arrays of integers a lot.  The reasoning is that dealing with small arrays
of integers is fast on modern systems; which is true.  There are, however, even
speedier alternatives.  Parts of the code could be using bit||sets to simulate
these arrays of integers.  Bit||set operations are generally several orders of
magnitude faster than operations on arrays of integers, and are in many cases
more space||efficient as well.  Bit||sets could be used in representing sets of
tags during compilation.  Bit||sets could also be used to keep track of active
states during execution, using methods based on those found in, e.g.,
\cite[Navarro01,Baeza-YatesNavarro].

Parameterized rules would perhaps add further expressitivity to the grammar,
allowing rules such as \TypedRegex{<line-containing(r)>} to be defined:

  \placeregex
    {none}
    \TypedRegex{<line-containing(r)> = ^.*<r>.*$}

The current implementation also lacks a syntax and semantics for symbol ranges.
Such a feature is dearly needed, but it has proven difficult to come up with a
satisfying definition.  This is an area where \Unicode\ support makes things
much more difficult, as one must commit to supporting so many more possible
expressions|/|ranges, and decide on so many more decisions on how such ranges
should be expressed.  All this has to be done while trying to keep things as
familiar as possible.  Symbol ranges are perhaps the most familiar part of
previous regular||expression syntaxes and changing them mercilessly may prove
to be too dangerous, if one is to attract people’s attention.

\Name{Ville}{Laurikari} has suggested a way to do approximate matching using
tags \cite[Laurikari01], which is used in his pattern||matching library.  The
method is straight||forward to implement, runs faster than previous algorithms,
and provides great flexibility in specifying the allowed Levenshtein||distances
\cite[Levenshtein66]\ (even on a per||transition basis).  The usefulness of
adding approximate matching remains to be seen; perhaps the best way to assess
its merit is simply to implement it and see\dots.

It would also be interesting to compare different methods of simulating finite
automata and to see how they perform.  There are many ways of doing so, and
each seems to have its merits.

\subsection{The Text Editor---\NED}

\NED\ still needs {\em a lot} of work before it can be considered usable by a
general public.  \NED\ desperately needs an interactive|/|visual user
interface.  Right now it only accepts commands on a command||line.  Other
features \NED\ is lacking include:

\startitemize
  \item Undo|/|Redo capabilities
  \item Syntax highlighting
  \item True \Unicode\ support
\stopitemize

The problem of undo|/|redo from a purely technical standpoint was briefly
discussed in \inchapter[text editors - internal].  And although such
functionality would be simple to implement, this may not be the desired
solution.  All text editors today perform undo|/|redo by subtracting and adding
to a sequential list of \Term{edit operations} representing the
\Term{edit history} of the buffer.  This works for well in most cases, but
doesn’t retain a complete history of the buffer, as it doesn’t keep track of
undo|/|redo operations.  To do this, a tree structure would have to be employed
instead, where new branches would be added for such operations.  This too is
rather simple to implement.  The big issue how to allow the user to get access
to this history.  How should the user be able to choose what branches to
traverse?  This question demands more thought before being answered.

Syntax highlighting is a method by which the text editor displays the contents
of the buffer using different colors and typefaces to show its user the
structure of the buffer.  Examples include highlighting comments in programming
languages in a distinct color or typeface, thus making it easier for the user
to separate comments from program code.  Syntax highlighting is a problem that
demands much thought.  The user has to be able to tell the text editor what to
highlight and in what manner to do so.  The text editor has to be able to keep
up with changes in the buffer and update its display accordingly---a single
character insertion may change the whole display!  Regular expressions are
often employed for the specification end of it \cite[Lesk75,Paxson95], so we
must make sure that our pattern||matching library is up for this task as well.

\Unicode\ is partially supported in \NED.  The implementation still makes some
assumptions about the input and doesn’t perform any translations from other
character sets to \Unicode\ before placing them in a buffer.  The problem
actually lies in Ruby’s implementation; Ruby currently doesn’t support
\Unicode.  The good news is that support is expected to be added before the end
of 2005.  Until then there are more pressing matters to attend to.

The buffer backend, the piece||tree data structure, is in need of some stress
tests to verify that it performs as well as we’ve hoped, considering its use of
space and time.

The command||line parser could also be made to accept user||defined tokens,
beyond the ability of accepting user||defined commands.  Such an addition would
warrant a rewrite of the whole command||line subsystem and should be well
thought||out before it was done.

The merit of using dependency injection in the implementation of \NED\ is also
up for debate.  After an initial flurry of great expectations from the new
pseudo||paradigm, interest has mostly died out, at least in its application on
Ruby.  It seems nice in theory, but may complicate matters unnecessarily in
practice.  It should, however, simplify unit testing.

And speaking of unit testing, both the pattern||matching library and \NED\ are
in need of unit tests.  Work has so far been done without giving much thought
to testing.  In the future, tests will be performed alongside development,
instead of being added at the end.

%\subsection{This Manuscript}
%
%It’s time to discuss possible improvements to this manuscript, and yes, there
%are some to be made.  The introduction to \Unicode\ in
%\insection[definition:character sets]\ should be extended to include further
%discussion of the transformations.  Some more diagrams to clarify how they work
%would also make for easier understanding.
%
%The structure of this document could also be made clearer.  Right now it’s kind
%of hard to tell exactly what has been done and why it was done.  A division into
%parts, i.e., a leveling above chapters, could be of some help.

\subsection{Miscellaneous}

It’s time to introduce ideas that aren’t bound to anything that has been
discussed in this manuscript already.

The first idea, which is one that’s been on my mind ever since I was introduced
to this formalism, is to try to harness the power of regular relations in a
text editor.  \Name{Markus}{Forsberg} suggested, during an initial session for
this master’s thesis, that I look into regular relations \cite[Kartunnen97].
To him, it seemed that many of the textual transformations that I had in mind
for my command||set for the text editor could be expressed using regular
relations, thus alleviating the separation between commands and regular
expressions in the command||set that I was proposing (and later used).  Time
and focus stopped me from pursuing this solution, which is kind of sad, as I
feel that I’ve lost a lot of time pondering the possibilities of such a
solution instead of focusing on the solution that I was actually implementing.

It’d be a great opportunity for me to be able to pursue the idea of using
regular relations in an interactive process, such as a text editor, to perform
textual transformations.  It’d also be a great opportunity for me to be able to
slap Xerox on the fingers for locking up one of the few tools available for
manipulating regular relations, \XFST \cite[Beesley03], behind ridiculous
licensing schemes, as I’d have a good excuse for writing such a library|/|tool
myself.  The challenge lies not so much in implementing the library itself,
rather it lies in coming up with a good syntax for expressing the many possible
operators that regular relations possess, see, e.g.,
\cite[Kartunnen96,Kempe96,Kartunnen97].

\stopcomponent
