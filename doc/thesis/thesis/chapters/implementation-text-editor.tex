\startcomponent thesis/chapters/implementation-text-editor

\project masters-project
\product thesis

\chapter
  [implementation - ned]
  {An Introduction to the \NED\ Text Editor}


In this chapter we’ll be introduced to the implementation of \NED, a new text
editor designed to make excellent use of regular expressions|<|of the \NRE\
variety|>|throughout it’s command||set.  This command||set has been greatly
inspired by that of the \SAM\ text editor and is currently limited to be but a
small superset of it.  In the future, useful commands found in other text
editors, such as \Vim\ and \EMACS, will surely be incorporated into the
command||set.  Nonetheless, the \SAM||based command||set provides the user with
many useful|<|and powerful|>|commands, and the use of {\NRE}s makes it easy to
use; a trait not inherited from its predecessors.

We start off with discussing the implementation of the buffering strategy
employed by \NED, the piece||table data structure introduced in
\insection[text editors - internal:piece table buffering strategy].  As the
table itself is represented using a red||black tree, we’ll henceforth refer to
it as a \Term{piece||tree}.

Once we’ve seen how piece||trees are implemented, we’ll continue with the
implementation of \NED\ itself.  \NED\ is written in Ruby, so the code will
look somewhat different.  But don’t fret, it’ll still be easy to follow, as it
is thoroughly discussed and simple to understand.

\startitemize
  \item Discuss the implementation of \NED.

  \item Introduce the implementation of the buffering strategy used, namely a
    piece table using a red-black tree for the second||level data structure.

  \item Discuss Ruby as an implementation language for the user interface and
    how to connect Ruby and C.

\stopitemize

\component thesis/sources/piecetree/piece.h.tex

\component thesis/sources/piecetree/piece.c.tex

\component thesis/sources/piecetree/node.h.tex

\component thesis/sources/piecetree/node.c.tex

\component thesis/sources/piecetree/iterator.h.tex

\component thesis/sources/piecetree/iterator.c.tex

\component thesis/sources/piecetree/piecetree.h.tex

\component thesis/sources/piecetree/piecetree.c.tex


\section{Implementing \NED}

The rest of this chapter is devoted to the implementation of \NED.  We’ll see
how the piece||tree data structure introduced in the preceeding section can be
used to represent a buffer and how to write a command||set that operates on
these buffers.  We’ll also see how to write a command||line parser so that we
may provide our user with a way to enter commands to be executed by the text
editor.


\component thesis/sources/ned/ned.rb.tex

\component thesis/sources/ned/ned/registry.rb.tex

\component thesis/sources/ned/ned/buffer.rb.tex

\component thesis/sources/ned/ned/buffer/buffer.rb.tex

\component thesis/sources/ned/ned/regexes.rb.tex

\component thesis/sources/ned/ned/command-line.rb.tex

\component thesis/sources/ned/ned/command-line/parsers.rb.tex

\component thesis/sources/ned/ned/command-line/parsers/scanner.rb.tex

\component thesis/sources/ned/ned/command-line/parsers/tokens.rb.tex

\component thesis/sources/ned/ned/command-line/parsers/simple.rb.tex

\component thesis/sources/ned/ned/command-line/parsers/tokens/command.rb.tex

\component thesis/sources/ned/ned/command-line/parsers/tokens/commands.rb.tex

\component thesis/sources/ned/ned/command-line/parsers/tokens/regex.rb.tex

\component thesis/sources/ned/ned/command-line/parsers/tokens/text.rb.tex

\component thesis/sources/ned/ned/command-line/commands.rb.tex

\component thesis/sources/ned/ned/command-line/command.rb.tex

\component thesis/sources/ned/ned/command-line/commands/append.rb.tex

\component thesis/sources/ned/ned/command-line/commands/change.rb.tex

\component thesis/sources/ned/ned/command-line/commands/insert.rb.tex

\component thesis/sources/ned/ned/command-line/commands/delete.rb.tex

\component thesis/sources/ned/ned/command-line/commands/extract.rb.tex

\component thesis/sources/ned/ned/command-line/commands/inject.rb.tex

\component thesis/sources/ned/ned/command-line/commands/substitute.rb.tex

\component thesis/sources/ned/ned/command-line/commands/guard.rb.tex

\component thesis/sources/ned/ned/command-line/commands/invertedguard.rb.tex

\component thesis/sources/ned/ned/command-line/commands/print.rb.tex

\component thesis/sources/ned/ned/command-line/commands/compound.rb.tex

% TODO: Discuss possible extensions that can easily be made using this
% strategy, such as syntax highlighting, undo|/|redo, and so on.

% TODO: Show some examples of using \NED.

\stopcomponent
