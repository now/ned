\startcomponent thesis/appendices/regex-editors

\project masters-project
\product thesis

\chapter{Glossary}

% TODO: add line||feed

% TODO: add meta||programming

\startgloss{big||endian}
In a \Term{big||endian} system, the most significant byte is stored first.
Thus, the number 1023 is stored as 00000000 00000000 00000011 11111111,
assuming a 32--bit system.
\stopgloss

\startgloss{byte}
A \Term{byte} is a logical unit in a computer system.  It represents eight bits
of information.
\stopgloss

\startgloss{\CLang}
A programming language written by \Name{Dennis M.}{Ritchie}.  Initially written
for for the \UNIX\ operating system as a way to enhance portability, \CLang\ is
now one of the most well known and used programming languages.  Its simplicity
and \quote{The Programmer is Always Right}||mindset makes it a favored choice
by many programmers.  There are, however, complaints about it being too much of
a low||level language, making many common tasks hard, or even unmaintainable to
do.

\Name{Brian W.}{Kernighan} is often credited as having invented the language
together with Dr. Ritchie, but has refuted such claims, saying that
\quotation{It's entirely Dennis Ritchie's work} \cite[LinuxJournal0703].  A
history of the language is given by Dr. Ritchie in \cite[Ritchie93].  The
reason for the mistaken attribution is most likely rooted in the fact that
Brian W. Kernighan is the main author of the most well known and respected book
on the subject, \PublicationTitle{The \CLang\ Programming Language}
\cite[KernighanRitchie88], together with Dr. Ritchie.

\CLang\ is covered in a couple of standards, the first defined by \ANSI\ in
1989 \cite[C89], then the standard was taken over by \ISO/\IEC\ in 1990
\cite[C90].  Under the 1990's, some effort was made to smoothly bring the
language up to date with current trends for making the entry to the 21st
century as smooth as possible, while keeping additions and changes to a
minimum.  The update is standardized by \ISO/\IEC\ in ISO/IEC 9899:1999
\PublicationTitle{Programming Languages. \CLang} \cite[C99].
\stopgloss

Desktop publishing, or DTP, is the process of editing and layout of printed
material intended for publication, such as books, magazines, brochures, and the
like using a personal computer. Desktop publishing software, such as
QuarkXPress, is software specifically designed for such tasks. Such programs do
not generally replace word processors and graphics applications, but are used
to aggregate content created in these programs: text, raster graphics (such as
images edited with Adobe Photoshop) and vector graphics (such as
drawings/illustrations made with Adobe Illustrator).

\startgloss{Desktop Publishing}
Desktop publishing|<|also referred to as \DTP|>|is the process of editing
books, newspapers, magazines, and so on, using a computer.  Desktop publishing
software can be considered to be the glue between many different kinds of
computer software, such as word processors, vector and raster graphic editors,
and so on.  The specific software used for the specific tasks may vary, but are
often bundled with the \DTP software itself.

One may argue that desktop publishing began with \Name{Donald E.}{Knuth}'s
\TeX\ software, which showed in 1978 that anyone with a (moderately sized at
the time) computer could become a publisher.  \TeX\ was placed in the public
domain which further spurred the advancement of high||quality printed material,
especially of a technical nature.

Examples of commercial applications are QuarkXPress, Adobe PageMaker, and
Microsoft Publisher, which|<|although costing a bundle|>|do not quite measure
up to the quality of Prof. Knuth's original software.
\stopgloss

% TODO: add reference to "The Unix Programming Environment" for this.
\startgloss{grep}
Initially a tool developed by Ken Thompson for searching files for matches of a
given regular expression, derived from the \type{g} command in the \Command{ed}
text editor.  The general form of the \type{g} command is \type{g/re/p}, where
\type{re} is a regular expression.  This command will print (\type{p}) every
line (\type{g}) matching the regular expression \type{re}.  Thus,
\DefineTerm{grep} actually stands for {\em General (search for) Regular
Expression and Print}.  Talk about having serious problems giving meaningful
names to your inventions.

Generally any tool that searches files for lines matching a given regular
expression.
\stopgloss

\startgloss{little||endian}
In a \Term{little||endian} system, the least significant byte is stored first.
Thus, the number 1023 is stored as 11111111 00000011 00000000 00000000,
assuming a 32--bit system.
\stopgloss

\startgloss{locale}
In computing, a set of parameters that define the user's language, country,
text encoding, time||zone, weekday names, month names, time and date formats,
eras, various kinds of punctuation (currency, decimal points, \dots), and so
on.
\stopgloss

\startgloss{\POSIX}
\POSIX\ \cite[POSIX92]\ is an international standard defining the operating
system interfaces for software designed for variants of the \UNIX\ operating
system.  It covers such things as system calls, \CLang\ header files, the
shell, various utilities/commands, among other things.
\stopgloss

\startgloss{word}
A \Term{word} is a logical unit in a computer system.  It represents whatever
number of bits that the computer architecture being discussed deals with for
random access memory, or \RAM, access.  Common word sizes are 16, 32, and 64.
\stopgloss

% TODO: fill in with more simple definitions that may be relevant
% (UTF-{8,16,32}, UTF, code point...bit and word?  script  code point
% byte order (BOM) LE BE UNIX

\stopcomponent
