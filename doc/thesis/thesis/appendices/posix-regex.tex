\startcomponent thesis/appendices/regex-editors

\project masters-project
\product thesis

\chapter
  [posix regex]
  {\POSIX\ Regular Expressions}


\section
  [posix regex.h]
  {The \POSIX\ \CLang\ Regular Expression Interface}

\Note This is the actual text of the section of the \POSIX\ \CLang\ interface
that discusses regular expressions, with some slight typographical and editor
changes to suit this manuscript.  Thus, this rendering should in no way be
considered official, nor should it be reproduced as such.

\subsection{Name}

{\tt regcomp}, {\tt regerror}, {\tt regexec}, {\tt regfree}---regular expression matching

\subsection{Synopsis}

\startC
#include <regex.h>

int regcomp(regex_t *restrict preg,
            const char *restrict pattern,
            int cflags);
size_t regerror(int errcode,
                const regex_t *restrict preg,
                char *restrict errbuf,
                size_t errbuf_size);
int regexec(const regex_t *restrict preg,
            const char *restrict string,
            size_t nmatch,
            regmatch_t pmatch[restrict],
            int eflags);
void regfree(regex_t *preg);
\stopC


\subsection{Description}

These functions interpret basic and extended regular expressions as described
in the Base Definitions volume of \IEEE\ Std 1003.1-2001, Chapter 9, Regular
Expressions.

The {\tt regex_t} structure is defined in {\tt <regex.h>} and contains at least
the following member:

\placetable
  []
  []
  {none}
  \starttable[|l|l|p(17em)|]
    \HL
    \NC \bf Member Type \NC \bf Member Name \NC \bf Description \NC\AR
    \HL
    \NC \tt size_t      \NC \tt re_nsub     \NC Number of parenthesized \crlf subexpressions. \NC\AR
    \HL
  \stoptable

The {\tt regmatch_t} structure is defined in {\tt <regex.h>} and contains at
least the following members:

\placetable
  []
  []
  {none}
  \starttable[|l|l|p(17em)|]
    \HL
    \NC \bf Member Type \NC \bf Member Name \NC \bf Description \NC\AR
    \HL
    \NC \tt regoff_t    \NC \tt rm_so\NC Byte offset from start of string to start of substring. \NC\AR
    \NC \tt regoff_t    \NC \tt rm_eo\NC Byte offset from start of string to the first character after the end of substring. \NC\AR
    \HL
  \stoptable

The ${\it regcomp}()$ function shall compile the regular expression contained
in the string pointed to by the ${\it pattern}$ argument and place the results
in the structure pointed to by ${\it preg}$. The ${\it cflags}$ argument is the
bitwise-inclusive \cap{or} of zero or more of the following flags, which are
defined in the {\tt <regex.h>} header:

\term{{\tt REG_EXTENDED}}
  Use Extended Regular Expressions.

\term{{\tt REG_ICASE}}
  Ignore case in match. (See the Base Definitions volume of \IEEE\ Std
  1003.1-2001, Chapter 9, Regular Expressions.)

\term{{\tt REG_NOSUB}}
  Report only success|/|fail in ${\it regexec}()$.

\term{{\tt REG_NEWLINE}}
  Change the handling of \<newline>s, as described in the text.

The default regular expression type for pattern is a Basic Regular Expression.
The application can specify Extended Regular Expressions using the {\tt
REG_EXTENDED} ${\it cflags}$ flag.

If the {\tt REG_NOSUB} flag was not set in ${\it cflags}$, then
${\it regcomp}()$ shall set ${\it re\_nsub}$ to the number of parenthesized
subexpressions (delimited by “\TypedRegex{\(\)}” in basic regular expressions
or “\TypedRegex{()}” in extended regular expressions) found in ${\it pattern}$.

The ${\it regexec}()$ function compares the null-terminated string specified by
${\it string}$ with the compiled regular expression ${\it preg}$ initialized by
a previous call to ${\it regcomp}()$. If it finds a match, ${\it regexec}()$
shall return 0; otherwise, it shall return non-zero indicating either no match
or an error. The ${\it eflags}$ argument is the bitwise-inclusive \cap{or} of
zero or more of the following flags, which are defined in the {\tt <regex.h>}
header:

\term{{\tt REG_NOTBOL}}
  The first character of the string pointed to by string is not the beginning
  of the line. Therefore, the circumflex character (‘\type{^}’), when taken
  as a special character, shall not match the beginning of string.

\term{{\tt REG_NOTEOL}}
  The last character of the string pointed to by string is not the end of the
  line. Therefore, the dollar sign (‘\type{$}’), when taken as a special
  character, shall not match the end of string.

If ${\it nmatch}$ is 0 or {\tt REG_NOSUB} was set in the ${\it cflags}$
argument to ${\it regcomp}()$, then ${\it regexec}()$ shall ignore the
${\it pmatch}$ argument.  Otherwise, the application shall ensure that the
${\it pmatch}$ argument points to an array with at least ${\it nmatch}$
elements, and ${\it regexec}()$ shall fill in the elements of that array with
offsets of the substrings of string that correspond to the parenthesized
subexpressions of ${\it pattern}$: ${\it pmatch}[i].{\it rm\_so}$ shall be the
byte offset of the beginning and ${\it pmatch}[i].{\it rm\_eo}$ shall be one
greater than the byte offset of the end of substring $i$. (Subexpression $i$
begins at the $i$th matched open parenthesis, counting from 1.) Offsets in
${\it pmatch}[0]$ identify the substring that corresponds to the entire regular
expression.  Unused elements of ${\it pmatch}$ up to
 ${\it pmatch}[{\it nmatch} - 1]$ shall be filled with $-1$. If there are more
than ${\it nmatch}$ subexpressions in pattern (${\it pattern}$ itself counts as
a subexpression), then ${\it regexec}()$ shall still do the match, but shall
record only the first ${\it nmatch}$ substrings.

When matching a basic or extended regular expression, any given parenthesized
subexpression of ${\it pattern}$ might participate in the match of several
different substrings of ${\it string}$, or it might not match any substring
even though the pattern as a whole did match. The following rules shall be used
to determine which substrings to report in ${\it pmatch}$ when matching regular
expressions:

\startitemize[n]
  \item[one] If subexpression $i$ in a regular expression is not contained
    within another subexpression, and it participated in the match several
    times, then the byte offsets in ${\it pmatch}[i]$ shall delimit the
    last such match.
  
  \item[two] If subexpression $i$ is not contained within another
    subexpression, and it did not participate in an otherwise successful match,
    the byte offsets in ${\it pmatch}[i]$ shall be $-1$. A subexpression does
    not participate in the match when:

    \startblockquote
      ‘\TypedRegex{*}’ or “\TypedRegex/\{\}/” appears immediately after the
      subexpression in a basic regular expression, or ‘\TypedRegex{*}’,
      ‘\TypedRegex{?}’, or “\TypedRegex/{}/” appears immediately after the
      subexpression in an extended regular expression, and the subexpression
      did not match (matched 0 times).
    \stopblockquote

    \noindent or:

    \startblockquote
          ‘\TypedRegex{|}’ is used in an extended regular expression to select
          this subexpression or another, and the other subexpression matched.
    \stopblockquote

  \item If subexpression $i$ is contained within another subexpression
    $j$, and $i$ is not contained within any other subexpression that
    is contained within $j$, and a match of subexpression $j$ is
    reported in ${\it pmatch}[j]$, then the match or non-match of
    subexpression $i$ reported in ${\it pmatch}[i]$ shall be as
    described in \in[one] and \in[two]. above, but within the substring
    reported in ${\it pmatch}[j]$ rather than the whole string. The offsets
    in ${\it pmatch}[i]$ are still relative to the start of string.

  \item If subexpression $i$ is contained within subexpression $j$, and the
    byte offsets in ${\it pmatch}[j]$ are $-1$, then the pointers in
    ${\it pmatch}[i]$ shall also be $-1$.

  \item If subexpression $i$ matched a zero-length string, then both of the
    byte offsets in ${\it pmatch}[i]$ shall be the byte offset of the character
    or null terminator immediately following the zero-length string.
\stopitemize

If, when ${\it regexec}()$ is called, the locale is different from when the
regular expression was compiled, the result is undefined.

If {\tt REG_NEWLINE} is not set in ${\it cflags}$, then a \<newline> in 
${\it pattern}$ or ${\it string}$ shall be treated as an ordinary character. If
{\tt REG_NEWLINE} is set, then \<newline> shall be treated as an ordinary
character except as follows:

\startitemize[n]
  \item A \<newline> in ${\it string}$ shall not be matched by a period outside
    a bracket expression or by any form of a non-matching list (see the Base
    Definitions volume of \IEEE\ Std 1003.1-2001, Chapter 9, Regular
    Expressions).

  \item A circumflex (‘\type{^}’) in ${\it pattern}$, when used to specify
    expression anchoring (see the Base Definitions volume of \IEEE\ Std
    1003.1-2001, Section 9.3.8, \BRE\ Expression Anchoring), shall match the
    zero-length string immediately after a \<newline> in ${\it string}$,
    regardless of the setting of {\tt REG_NOTBOL}.

  \item A dollar sign (‘\type{$}’) in pattern, when used to specify
    expression anchoring, shall match the zero-length string immediately before
    a \<newline> in string, regardless of the setting of {\tt REG_NOTEOL}.
\stopitemize

The ${\it regfree}()$ function frees any memory allocated by ${\it regcomp}()$
associated with ${\it preg}$.

The following constants are defined as error return values:

\term{{\tt REG_NOMATCH}}
  ${\it regexec}()$ failed to match.

\term{{\tt REG_BADPAT}}
  Invalid regular expression.

\term{{\tt REG_ECOLLATE}}
  Invalid collating element referenced.

\term{{\tt REG_ECTYPE}}
  Invalid character class type referenced.

\startterm{{\tt REG_EESCAPE}}
  Trailing ‘\TypedRegex{\}’ in pattern.
\stopterm

\term{{\tt REG_ESUBREG}}
  Number in “\<digit>” invalid or in error.

\startterm{{\tt REG_EBRACK}}
  “\type{[]}” imbalance.
\stopterm

\startterm{{\tt REG_EPAREN}}
  “\TypedRegex{\(\)}” or “\TypedRegex{()}” imbalance.
\stopterm

\startterm{{\tt REG_EBRACE}}
  “\TypedRegex/\{\}/” imbalance.
\stopterm

\startterm{{\tt REG_BADBR}}
  Content of “\TypedRegex/\{\}/” invalid: not a number, number too large,
  more than two numbers, first larger than second.
\stopterm

\term{{\tt REG_ERANGE}}
  Invalid endpoint in range expression.

\term{{\tt REG_ESPACE}}
  Out of memory.

\startterm{{\tt REG_BADRPT}}
  ‘\TypedRegex{?}’, ‘\TypedRegex{*}’, or ‘\TypedRegex{+}’ not preceded by
  valid regular expression.
\stopterm

The ${\it regerror}()$ function provides a mapping from error codes returned by
the ${\it regcomp}()$ and ${\it regexec}()$ functions to unspecified printable
strings. It generates a string corresponding to the value of the ${\it
errcode}$ argument, which the application shall ensure is the last non-zero
value returned by ${\it regcomp}()$ or ${\it regexec}()$ with the given value
of ${\it preg}$. If ${\it errcode}$ is not such a value, the content of the
generated string is unspecified.

If ${\it preg}$ is a null pointer, but ${\it errcode}$ is a value returned by a
previous call to ${\it regexec}()$ or ${\it regcomp}()$, the ${\it regerror}()$
still generates an error string corresponding to the value of ${\it errcode}$,
but it might not be as detailed under some implementations.

If the ${\it errbuf\_size}$ argument is not 0, ${\it regerror}()$ shall place
the generated string into the buffer of size ${\it errbuf\_size}$ bytes pointed
to by ${\it errbuf}$. If the string (including the terminating null) cannot fit
in the buffer, ${\it regerror}()$ shall truncate the string and null-terminate
the result.

If ${\it errbuf\_size}$ is 0, ${\it regerror}()$ shall ignore the errbuf
argument, and return the size of the buffer needed to hold the generated
string.

If the ${\it preg}$ argument to ${\it regexec}()$ or ${\it regfree}()$ is not a
compiled regular expression returned by ${\it regcomp}()$, the result is
undefined. A ${\it preg}$ is no longer treated as a compiled regular expression
after it is given to ${\it regfree}()$.

\subsection{Return Value}

Upon successful completion, the ${\it regcomp}()$ function shall return 0.
Otherwise, it shall return an integer value indicating an error as described in
{\tt <regex.h>}, and the content of ${\it preg}$ is undefined. If a code is
returned, the interpretation shall be as given in {\tt <regex.h>}.

If ${\it regcomp}()$ detects an invalid \RE, it may return {\tt REG_BADPAT}, or
it may return one of the error codes that more precisely describes the error.

Upon successful completion, the ${\it regexec}()$ function shall return 0.
Otherwise, it shall return {\tt REG_NOMATCH} to indicate no match.

Upon successful completion, the ${\it regerror}()$ function shall return the
number of bytes needed to hold the entire generated string, including the null
termination. If the return value is greater than ${\it errbuf\_size}$, the
string returned in the buffer pointed to by ${\it errbuf}$ has been truncated.

The ${\it regfree}()$ function shall not return a value.

\subsection{Errors}

No errors are defined.

\stopcomponent
